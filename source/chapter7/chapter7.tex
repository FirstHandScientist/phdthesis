\chapter{Powering Hidden Markov Model by Normalizing Flows}
\label{chpt7:genhmm}
We have been mainly discussing the topics of modeling and learning where an observation instance is independent of others so far. In another word, the assumption of independent and identically distributed observation instances has been used. In this chapter, we extend to the topic of modeling and learning for sequential or temporal signals, such as speech signal, trajectories of a robot's movement, DNA sequence, etc.

In modeling a dynamic system that generates sequential signals, we are usually interested in reasoning about the system state that evolves over time underlining the signal. For simplification, the timeline over which a dynamic system generates sequential signal is discretized, i.e. time is sliced. Thus for each time slice $t$, we can take a measurement of the dynamic system, which corresponds to an observed variable $\bm{x}_t$. Due to the limitation of our measure or the abstraction in modeling itself, the state of the system at this time slice is not directly available, which corresponds to an unobserved or hidden variable $s_t$.

One assumption change makes a big difference. The straightforward issue is the complexity of the graphical representation of dependency, which in turn affects model learning and inference. As the dynamic system evolves, the dependency (correspondence to edges in the graphical model) can be arbitrarily complex. The complexity can be too high for practical usages in general. Therefore constraints are required to allow reasonable model learning and state estimations.
The most widely used constraint is probably the Markov assumption, i.e. $(\bm{x}_{t+1}, {s}_{t+1})$ is independent of $(\bm{x}_{1:t-1}, {s}_{1:t-1})$ if $(\bm{x}_{t}, {s}_{t})$ is given. This assumption indeed reduces the complexity of graphical structure in modeling the dynamic system, since we would not need to draw any edges between variables with time interval larger than one, in graphical representation.

The other issue is the parameterization of the model of the dynamic system. If we directly model $\{\bm{x}_t, {s}_t, t=1, 2, \cdots, T\}$ jointly, the parameterization would grow exponentially as the system evolves over time. A good solution to this issue is to introduce the \textit{template} concept into the graphical model. In the assumption, the variable $\bm{x}_t$ or $s_t$ becomes an instance of a \textit{template variable}. More importantly, each dependency between two continuous time slices becomes an instance of a \textit{template factor}. When the dynamic evolves from time $t$ to $t+1$, we only need to instantiate from the template variables and template factors without adding new parameters to the model. Template based dynamic Bayesian networks belong to such kind of models.

We add one more assumption that variable $\bm{x}_t$ is independent of the rest of variables $\{\bm{x}_{t^{\prime}}, {s}_{t^{\prime}}, t^{\prime}\neq t \}$ if the state $s_t$ is given, we reduce a dynamic Bayesian network into the classic hidden Markov model (HMM). Similar to chapter~\ref{chpt6:em-flow}, we bring the normalizing flows into the \textit{template factors} modeling to increase the flexibility of HMM in modeling dynamic systems in this chapter.

\section{Hidden Markov Model}
\begin{figure}[tp!]
  \centering
  \begin{subfigure}{0.35\textwidth}
  \begin{tikzpicture}
    \tikzstyle{enode} = [thick, draw=black, circle, align=center]
    \tikzstyle{cnode} = [thick, draw=black, circle, align=center, inner sep = 0.3pt]
    \tikzstyle{nnode} = [thick, rectangle, rounded corners = 2pt,minimum size = 0.8cm,draw,inner sep = 8pt]
    \node[enode] (s) at (-1.5,2) {${s}$};
    \node[enode] (x) at (0,0){$\bm{x}^{\prime}$};
    \node[enode] (sp) at (0,2){${s}^{\prime}$};
    \node[nnode, fit=(x)(sp)(s)] (box) {};
    % \node[] at (-1.8,-0.4) {$\abs{\Dd}$};
    
    \draw[->] (s) to (sp);
    \draw[->] (sp) to (x);

  \end{tikzpicture}
  \caption{The template model of HMM.}\label{chpt7:fig:hmm-template}
\end{subfigure}
\hspace{5pt}
\begin{subfigure}{0.45\textwidth}
  \begin{tikzpicture}
    \tikzstyle{enode} = [thick, draw=black, circle, align=center]
    \tikzstyle{cnode} = [thick, draw=black, circle, align=center, inner sep = 0.3pt]
    \tikzstyle{nnode} = [thick, rectangle, rounded corners = 2pt,minimum size = 0.8cm,draw,inner sep = 8pt]
    \node[enode] (s0) at (0,2) {${s}_0$};
    \node[enode] (s1) at (1.5,2){${s}_1$};
    \node[enode] (s2) at (3,2){${s}_2$};
    \node[enode] (s3) at (4.5,2){${s}_3$};
    
    \node[enode] (x1) at (1.5,0){$\bm{x}_1$};
    \node[enode] (x2) at (3,0){$\bm{x}_2$};
    \node[enode] (x3) at (4.5,0){$\bm{x}_3$};
    \node[nnode, draw=white, fit=(s0)(s1)(x1)] (box) {};
    
    \draw[->] (s0) to (s1);
    \draw[->] (s1) to (s2);
    \draw[->] (s2) to (s3);

    \draw[->] (s1) to (x1);
    \draw[->] (s2) to (x2);
    \draw[->] (s3) to (x3);
    
  \end{tikzpicture}
  % \vskip 10pt
  \caption{The unrolled HMM instance to time $t=3$.}\label{chpt7:fig:hmm-instance}
\end{subfigure}
\caption{From HMM template to instance.}\label{chpt7:fig:hmm-inituition}
\end{figure}

We detail the concept of Markov assumption and \textit{template} with an example before giving the definition of HMM. As shown in Figure~\ref{chpt7:fig:hmm-template}, the plate denotes the template within which the generic variables and their dependencies are represented. As a dynamic evolves with time, the hidden variable $s$ and observed variable $\bm{x}$ can be instantiated for each time slice $t$, along with their dependencies. Figure~\ref{chpt7:fig:hmm-instance} is an instantiated example till $t=3$ from the generic template. The Markov assumption is embedded in the HMM instances from the template since Markov independence is fulfilled no matter how long the system evolves. More importantly, the conditional probabilities $p(s_{t+1}|s_{t}),~\forall~t$, instantiated from template factors (defined with the directed edge $s\rightarrow s^{\prime}$ in Figure~\ref{chpt7:fig:hmm-template}), share the same parameterization. Similarly, the probabilities $p(\bm{x}_{t}|s_{t}),~\forall~t$ share parameterization with the template factor defined with edge $s^{\prime}\rightarrow \bm{x}^{\prime}$.

With the intuition from Figure~\ref{chpt7:fig:hmm-inituition}, we now define the HMM with its parameterization. A HMM $\bm{H}$ defined in a hypothesis space $\Hh$, i.e. $\bm{H} \in \Hh$, is capable to model time-span signal $\ubar{\bm{x}} = \left[ \bm{x}_1, \cdots, \bm{x}_T\right]^{\intercal}$, where $\bm{x}_t\in \RR^{N}$ is the $N$-dimensional signal at time $t$, $[\cdot]^{\intercal}$ denotes transpose, and $T$ denotes the time length\footnote{The length for  sequential data varies.}. We define the hypothesis set of HMM as $\Hh := \{\bm{H} | \bm{H}=\{\Ss, \bm{q}, \bm{A}, p(\bm{x}|{s}; \bm{\Phi}_{s})\}\}$, where
\begin{itemize}
\item $\Ss$ is the set of hidden states of $\bm{H}$.
\item $\bm{q} = \left[ q_1, q_2, \cdots, q_{|\Ss|}\right]^\intercal$ is the initial state distribution of $\bm{H}$ with $|\Ss|$ as cardinality of $\Ss$. For $i \in \Ss$, $q_i = p(s_{1}=i;\bm{H})$. We use $s_t$ to denote the state $s$ at time $t$.
\item $\bm{A}$ matrix of size $|\Ss| \times |\Ss|$ is the transition matrix of states in $\bm{H}$. That is, $\forall i, j \in \Ss$,  $\bm{A}_{i,j} = p(s_{t+1}=j|s_{t}=i; \bm{H})$.
\item For a given hidden state $s$, the density function of the observable signal is $p({\bm{x}}|{s};\bm{\Phi}_{s})$, where $\bm{\Phi}_{s}$ is the parameter set that defines this probabilistic model. Denote $\bm{\Phi} = \left\{ \bm{\Phi}_{s}| s \in \Ss \right\}$.
\end{itemize}


\begin{figure}[!t]
  \centering
  \begin{tikzpicture}
    \tikzstyle{enode} = [thick, draw=black, ellipse, inner sep = 1pt,  align=center]
    \tikzstyle{nnode} = [thick, rectangle, rounded corners = 2pt,minimum size = 0.8cm,draw,inner sep = 2pt]
    \node[enode] (g1) at (-0.5,1.8) {$p(\bm{x}| s=1; \bm{\Phi}_{1})$};
    \node[enode] (g2) at (-0.5,0.5) {$p(\bm{x}| s=2; \bm{\Phi}_{2})$};
    \node[enode] (gs) at (-0.5, -1.8) {$p(\bm{x}| s=|\Ss|; \bm{\Phi}_{|\Ss|})$};
    \node[enode] (x) at (4.5,1.5){$\ubm{x}\sim p(\ubm{x};\bm{H})$};

    \draw[dotted,line width=2pt] (0,-0.3) -- (0,-1.2);
    \filldraw[->] (1.9, 0.5)circle (2pt) --  (x) ;
    \draw[->] (g1) -- (1.8, 1.8);
    \draw[->] (g2) -- (1.8, 0.5);
    \draw[->] (gs) -- (1.8, -1.8);

    \begin{scope}[xshift=0.5cm, thick, every node/.style={sloped,allow upside down}]
      \node[nnode] (m) at (3.5,-2) {Memory};
      \node[nnode] (a) at (3.5,-0.5) {$\bm{A}$};

      \draw (2.1,0.9)-- (2.2, 0.);
      \draw (2.2,0.)-- node {\midarrow} (2.2,-2);
      \draw (2.2,-2)-- (m);
      \draw (m)-- (5, -2);
      \draw (5, -2)-- node {\midarrow} (5 ,-0.5);
      \draw (5, -0.5) -- (a);
      \draw (a)-- node {\midarrow} (2.2, -0.5);
      \node at (4.8, -1) {$s_{t}$};
      \node at (2.56, -0.25) {$s_{t+1}$};
    \end{scope}
  \end{tikzpicture}
  \caption{HMM model: a generative illustration.}\label{fig:hmm}
  \vspace{0.3cm}
\end{figure}


Using HMM for signal representation is illustrated in Figure~\ref{fig:hmm}. The model assumption is that different instant signal of $\ubar{\bm{x}}$ is generated by a different signal source associated with a hidden state of HMM.
In the framework of HMM, at each time instance $t$, signal $\bm{x}_t$ is assumed to be generated by a distribution with density function $p(\bm{x}_t| s_t; \bm{\Phi}_{s_t})$, and $s_t$ is decided by the hidden markov process. Putting these together gives us the probabilistic model $p(\ubm{x};\bm{H})$.

\begin{remark}
The graphical model of an HMM is very similar to that of a conditional random field (CRF), especially linear-chain CRF. The straightforward difference between an HMM and a linear-chain CRF is the graphical model representation. An HMM is a pure directed probabilistic graphical model while there are both directed and undirected edges in a CRF. In a nutshell, omitting observed variables and their variables, the rest variables and their edges forms an MRF in a CRF. This difference also affects the model training. An HMM is a generative model and models $p(\ubar{\bm{x}}, \ubar{{s}})$ jointly. Its training target is to maximize the joint probability. In contrast, a CRF models a conditional distribution $p(\ubar{{s}}|\ubar{\bm{x}})$ and is learned via discriminative training. See \cite{charles2012crf} for a more detailed discussion on the difference between them.
\end{remark}

\section{Generator-mixed HMM (GenHMM)}


\subsection{Generative Model of GenHMM}

\begin{figure}[!t]
  \centering
  \begin{tikzpicture}
    \tikzstyle{enode} = [thick, draw=black, ellipse, inner sep = 2pt,  align=center]
    \tikzstyle{nnode} = [thick, rectangle, rounded corners = 2pt,minimum size = 0.8cm,draw,inner sep = 2pt]
    \node[enode] (z1) at (-1.2,1.8) {$\bm{z}\sim p_{s,1}(\bm{z})$};
    \node[nnode] (g1) at (1,1.8) {$\bm{g}_{s,1}$};
    \node[enode] (z2) at (-1.2,0.5){$\bm{z}\sim p_{s,2}(\bm{z})$};
    \node[nnode] (g2) at (1,0.5) {$\bm{g}_{s,2}$};
    \node[enode] (zK) at (-1.2,-1.8) {$\bm{z}\sim p_{s,K}(\bm{z})$};
    \node[nnode] (gs) at (1, -1.8) {$\bm{g}_{s,K}$};
    \node[enode] (x) at (4.5,0){$\bm{x}\sim p(\bm{x}| s; \bm{\Phi}_{s})$};

    \draw[dotted,line width=2pt] (0,-0.3) -- (0,-1.2);
    \filldraw[->] (1.9, 0.5)circle (2pt) --  node[above=0.2]{${\kappa}\sim \bm{\pi}_{s}$} (x)  ;
    \draw[->] (z1) -- (g1);
    \draw[->] (g1) -- (1.8, 1.8);

    \draw[->] (z2) -- (g2);
    \draw[->] (g2) -- (1.8, 0.5);

    \draw[->] (zK) -- (gs);
    \draw[->] (gs) -- (1.8, -1.8);
  \end{tikzpicture}
  \caption{Template factor of GenHMM: conditional probabilistic model of observed variable $\bm{x}$ given hidden state $s$.}
  \label{fig:gen-mix}
  \vspace{0.3cm}
\end{figure}

In this section, we introduce normalizing flows to the state probabilistic model of our GenHMM, which models the conditional probability of observation given hidden state.
Recall that $\bm{x}\in\mathbb{R}^N$. Subscript is omitted when it does not cause ambiguity.
The probabilistic model of GenHMM for each hidden state is a mixture of $K$ flow generators that are implemented by neural networks, where $K$ is a positive integer.
The probabilistic model of a state $s\in\Ss$ is then given by
\begin{equation}\label{eq:state-prob-model}
  p(\bm{x}| s; \bm{\Phi}_{s}) = \sum_{\kappa=1}^{K}\pi_{s, \kappa} p(\bm{x}| s, \kappa; \bm{\theta}_{s, \kappa}),
\end{equation}
where $\kappa$ is a random variable following a categorical distribution, with probability $\pi_{s, \kappa} = p(\kappa | s; \bm{H})$.
Naturally $\sum_{\kappa = 1}^{K} \pi_{s, \kappa}= 1$. Denote $\bm{\pi}_{s} = [\pi_{s,1}, \pi_{s,2}, \cdots, \pi_{s,K}]^{\intercal}$. 
In \eqref{eq:state-prob-model}, $p(\bm{x}| s, \kappa; \bm{\theta}_{s, \kappa})$ is defined as induced distribution by a generator $\bm{g}_{s,\kappa}: \RR^{N}\rightarrow\RR^{N}$, such that $\bm{x}=\bm{g}_{s, \kappa}(\bm{z})$, where $\bm{z}$ is a latent variable following a distribution with density function $p_{s,\kappa}(\bm{z})$. Generator $\bm{g}_{s,\kappa}$ is parameterized by $\bm{\theta}_{s, \kappa}$. Let us denote the collection of the parameter sets of generators for state $s$ as $\bm{\theta}_s = \left\{ \bm{\theta}_{s, \kappa}| \kappa = 1, 2, \cdots, K \right\}$. For a flow generator $\bm{g}_{s, \kappa}$, we have
\begin{equation}\label{chpt7:eq:change-variable}
  p(\bm{x}| s, \kappa; \bm{\theta}_{s, \kappa}) = p_{s,\kappa}(\bm{z})\bigg| \det\left( \pd{\bm{g}_{s,\kappa}(\bm{z})}{\bm{z}} \right)\bigg|^{-1}.
\end{equation}

The signal of the probability distribution for a state $s$ of GenHMM is shown in Figure~\ref{fig:gen-mix}, in which the generator identity is up to the random variable $\kappa$. This serves as the template factor in instantiating the chain of a HMM. The normalizing flows with neural network realization of their coupling mapping \eqref{chpt6:eq-gl-coupling} allow richer probability density function space and offer higher model expressivity. This naturally enriches the feasible space $\Hh$. Putting these together with the typical Markov assumption, the template model of GenHMM is illustrated in Figure~\ref{chpt7:fig:tenhmm-template}. As a dynamic system evolves, we can instantiate Figure~\ref{chpt7:fig:tenhmm-template} to create the dynamic chain of graphical model representation.

\subsection{Learning in EM framework}

Assume the sequential signal $\ubm{x}$ follows the true distribution $p^{\prime}(\ubm{x})$, which is unknown. We would like to use GenHMM to model this distribution. Alternatively, we are looking for the answer to the question
\begin{equation}
  \umin{\bm{H}\in \Hh} KL({p^{\prime}}(\ubm{x})\| p(\ubm{x};\bm{H})),
\end{equation}
where $KL(\cdot\|\cdot)$ denotes the Kullback-Leibler divergence. For practical consideration, we only have access to the samples of $p^{\prime}(\ubm{x})$, i.e. the dataset of this distribution. For the given dataset, we denote its empirical distribution by $\hat{p}(\ubm{x}) = \frac{1}{R}\sum_{r=1}^{R} \delta_{\ubmr{x}{r}}(\ubm{x})$, where $R$ denotes the total number of sequential samples and superscipt $(\cdot)^{r}$ denotes the index of $r$-th sequential signal. 
The KL divergence minimization problem can be reduced to a likelihood maximization problem
\begin{equation}\label{eq:ml-of-hmm}
  \uargmax{\bm{H} \in \Hh} \frac{1}{R}\sum_{r=1}^{R}\log\,p(\ubmr{x}{r}; \bm{H}).
\end{equation}

\begin{figure}[tp!]
  \centering
  \begin{tikzpicture}
    \tikzstyle{enode} = [thick, draw=black, circle, align=center]
    \tikzstyle{cnode} = [thick, draw=black, circle, align=center, inner sep = 0.3pt]
    \tikzstyle{nnode} = [thick, rectangle, rounded corners = 2pt,minimum size = 0.8cm,draw,inner sep = 8pt]
    \node[enode] (s) at (-1.5,-1) {${s}$};
    \node[enode] (k) at (-1.5,1) {${\kappa}$};
    \node[enode] (z) at (-1.5,0) {$\bm{z}$};
    
    \node[enode] (sp) at (0,-1){${s}^{\prime}$};
    \node[enode] (kp) at (0,1){${\kappa}^{\prime}$};
    \node[enode] (zp) at (0,0){$\bm{z}^{\prime}$};
    
    \node[enode] (x) at (1.4,-2){$\bm{x}^{\prime}$};
    
    \node[nnode, fit=(x)(z)(k)(s)] (box) {};
    % \node[] at (-1.8,-0.4) {$\abs{\Dd}$};
    
    \draw[->] (s) to (sp);
    \draw[->] (sp) to (x);
    \draw[->] (zp) to (x);
    \draw[->] (kp) to (x);
  \end{tikzpicture}
  \caption{The template model of GenHMM.}\label{chpt7:fig:tenhmm-template}
\end{figure}
For the likelihood maximization, the first problem that we need to address is to deal with the hidden sequential variables of model $\bm{H}$, namely $\ubm{s}=[ \bm{s}_1, \bm{s}_2, \cdots, \bm{s}_T ]^{\intercal}$ and $\ubm{\kappa} = [\bm{\kappa}_1, \bm{\kappa}_2, \cdots, \bm{\kappa}_T]^{\intercal}$. For a sequential observable variable $\ubm{x}$, $\ubm{s}$ is the hidden state sequence corresponding to $\ubm{x}$, and $\ubm{\kappa}$ is the hidden variable sequence representing the generator identity sequence that actually generates $\ubm{x}$. Note that for a sequential observable $\ubm{x}$, there is also a sequential $\ubm{z}$ brought by flows. But mapping between $\bm{x}$ and $\bm{z}$ is deterministic for fixed $p_{s,\kappa}(\bm{z})$ when the corresponding $\bm{s}$ and ${\kappa}$ are given, which is defined by $\bm{g}_{s, \kappa}$.

Since directly maximizing likelihood is not an option for our problem in \eqref{eq:ml-of-hmm}, we address this problem with EM framework. This divides our problem into  two iterative steps: i) using the joint posterior of hidden variable sequences $\ubm{s}$ and $\ubm{\kappa}$ to obtain an ``expected likelihood'' of the observable variable sequence $\ubm{x}$, i.e. the E-step; ii) maximizing the expected likelihood w.r.t. the model $\bm{H}$, i.e. the M-step. Assume model $\bm{H}$ is at a configuration of $\bm{H}^{\mathrm{old}}$, we formulate these two steps as follows.
\begin{itemize}
\item E-step: % the posterior probability of $\ubm{s}$:
  % \begin{equation}
  %   p(\ubm{s}|\ubm{x})
  % \end{equation}
  the expected likelihood function
  \begin{equation}\label{eq:em-q-funciton}
    \Qq(\bm{H}; \bm{H}^{\mathrm{old}}) = \EE_{\hat{p}(\ubm{x}),p(\ubm{s},\ubm{\kappa}| \ubm{x}; \bm{H}^{\mathrm{old}})}\left[ \log\,p(\ubm{x}, \ubm{s}, \ubm{\kappa}; \bm{H})\right],
  \end{equation}
  where $\EE_{\hat{p}(\ubm{x}),p(\ubm{s},\ubm{\kappa}| \ubm{x}; \bm{H}^{\mathrm{old}})}\left[ \cdot\right]$ denotes the expectation operator by distribution $\hat{p}(\ubm{x})$ and $p(\ubm{s},\ubm{\kappa}| \ubm{x}; \bm{H}^{\mathrm{old}})$.
\item M-step: the maximization step
  \begin{equation}\label{eq:em-m-opt}
    \umax{\bm{H}} \Qq(\bm{H}; \bm{H}^{\mathrm{old}}).
  \end{equation}
\end{itemize}


The problem \eqref{eq:em-m-opt} can be reformulated as
\begin{align}\label{eq:m-step-subs}
  \umax{\bm{H}} \Qq(\bm{H}; \bm{H}^{\mathrm{old}})
  =\umax{\bm{q}}\Qq(\bm{q}; \bm{H}^{\mathrm{old}}) + \umax{\bm{A}}\Qq(\bm{A}; \bm{H}^{\mathrm{old}}) 
     + \umax{\bm{\Phi}}\Qq(\bm{\Phi}; \bm{H}^{\mathrm{old}}),
\end{align}
where the decomposed optimization problems are
\begin{align}
  \Qq(\bm{q}; \bm{H}^{\mathrm{old}}) 
  % &=\EE_{\hat{p}(\ubm{x}),p(\ubm{s},\ubm{\kappa}| \ubm{x}; \bm{H}^{\mathrm{old}})} \left[ \log\,p({s}_{1})  \right] \nonumber\\
    &= \EE_{\hat{p}(\ubm{x}),p(\ubm{s}| \ubm{x}; \bm{H}^{\mathrm{old}})} \left[ \log\,p({s}_{1};\bm{H})  \right], \label{eq:init-distribution-update}\\
  \Qq(\bm{A}; \bm{H}^{\mathrm{old}}) &=\EE_{\hat{p}(\ubm{x}),p(\ubm{s}| \ubm{x}; \bm{H}^{\mathrm{old}})}\hspace{-0.1cm}\left[ \sum_{t=1}^{T-1}\log\,p({s}_{t+1}|{s}_{t}; \bm{H}) \right], \label{eq:transition-update}\\
  \Qq(\bm{\Phi}; \bm{H}^{\mathrm{old}}) &= \EE_{\hat{p}(\ubm{x}),p(\ubm{s},\ubm{\kappa}| \ubm{x}; \bm{H}^{\mathrm{old}})} \left[ \log\,p(\ubm{x}, \ubm{\kappa}| \ubm{s}; \bm{H}) \right]. \label{eq:generative-model-update}
\end{align}

We can see that the solution of $\bm{H}$ depends on the posterior probability $p(\ubm{s}| \ubm{x}; \bm{H})$. Though the evaluation of posterior according to Bayes theorem is straightforward, the computation complexity of $p(\ubm{s}| \ubm{x}; \bm{H})$ grows exponentially with the length of $\ubm{s}$. Therefore, we employ forward-backward algorithm \cite{Bishop:2006:PRM:1162264} to do the posterior computation efficiently. As we would detail in the next section, what are needed to formulate the problem, are actually the $p(s| \ubm{x}; \bm{H})$ and $p(s, \kappa| \ubm{x}; \bm{H})$. For the joint posterior $p(s, \kappa| \ubm{x}; \bm{H})$, it can be computed by the Bayes rule when posterior of hidden state is available.


With such a solution framework ready, we detail the practical learning algorithm for GenHMM with normalizing flows embedded in next section.

\section{Practical Solution to GenHMM}

In this section, we detail the solution for realizing and learning GenHMM. % We begin with introducing flow model that we are going to use to model our generators $\{\bm{g}_{s, \kappa}| s\in\Ss, \kappa=1,2,\cdots, K\}$ for efficient likelihood computation. The learning method of HMM is followed then. 
The convergence of GenHMM is also discussed in this section.


\subsection{Realizing $\bm{g}_{s,\kappa}$ by a Flow Model}
Each generator $\bm{g}_{s,\kappa}$ is realized as a feed-forward neural netowrk.
We define generator $\bm{g}_{s,\kappa}$ as a $L$-layer flow model and formulate its mapping by layer-wise concatenation:
\begin{equation}
\bm{g}_{s,\kappa}=\bm{g}_{s,\kappa}^{[L]}\circ \bm{g}_{s,\kappa}^{[L-1]}\circ \cdots \circ \bm{g}_{s,\kappa}^{[1]},
\end{equation}
where superscript $[l]$ denotes the layer index and $\circ$ denotes mapping concatenation. As detailed in section~\ref{chpt6:sec:flow}, generator $\bm{g}_{s,\kappa}$ is invertible and denote its inverse mapping as $\bm{f}_{s,\kappa}=\bm{g}_{s,\kappa}^{-1}$. Then \eqref{chpt7:eq:change-variable} can be rewritten as
\begin{equation}\label{chpt7:eq:change-variable-flow}
  \log{p(\bm{x}| s, \kappa; \bm{\theta}_{s, \kappa})} = \log{p_{s,\kappa}(\bm{f}_{s, \kappa}(\bm{x}))} + \log{\bigg| \det\left( \pd{\bm{f}_{s,\kappa}(\bm{x})}{\bm{z}} \right)\bigg|}.
\end{equation}
By decomposing the flow model into layer-wise mapping, the part of Jacobian matrix determinant becomes
\begin{equation}\label{eq:cat-jacobian}
  \begin{array}{rl}
    \mathrm{det}(\nabla{\bm{f}_{s,\kappa}}) = \prod_{l=1}^L \det (\nabla{\bm{f}_{s,\kappa}^{[l]}}),
  \end{array}
\end{equation}
where $\nabla{\bm{f}_{s,\kappa}^{[l]}}$ is the Jacobian of the mapping from the $l$-th layer to the $(l-1)$-th layer, i.e., the inverse transformation.
Then
\begin{equation}\label{chpt7:eq:change-variable-flow-layer}
  \log{p(\bm{x}| s, \kappa; \bm{\theta}_{s, \kappa})} = \log{p_{s,\kappa}(\bm{f}_{s, \kappa}(\bm{x}))} + \sum_{l=1}^{L}\log{\bigg| \det\left(\nabla{\bm{f}_{s,\kappa}^{[l]}}\right)\bigg|}.
\end{equation}


\subsection{Learning of GenHMM}\label{subsec:optmGenHMM}
In this subsection, we address the problem of learning GenHMM.
\subsubsection{Generative Model Learning}
The generative model learning is actually to solve the problem in \eqref{eq:generative-model-update}, which can be further divided into two subproblems: i) generator learning; ii) mixture weights of generators learning. Let us define notations:
$\bm{\Pi} = \left\{  \bm{\pi}_{s}| s\in \Ss \right\}$, $\bm{\Theta}=\left\{ \bm{\theta}_s| s\in \Ss \right\}$. 
Then the problem in \eqref{eq:generative-model-update} becomes
\begin{align}\label{eq:sub-gm}
  &\umax{\bm{\Phi}} \Qq(\bm{\Phi}; \bm{H}^{\mathrm{old}}) = \umax{\bm{\Pi}} \Qq(\bm{\Pi}; \bm{H}^{\mathrm{old}}) + \umax{\bm{\Theta}} \Qq(\bm{\Theta}; \bm{H}^{\mathrm{old}}),
\end{align}
where
\begin{align}
  \Qq(\bm{\Pi}; \bm{H}^{\mathrm{old}})  &=\EE_{\hat{p}(\ubm{x}),p(\ubm{s},\ubm{\kappa}| \ubm{x}; \bm{H}^{\mathrm{old}})}\left[  \log\,p(\ubm{\kappa}| \ubm{s}; \bm{H})\right], \\
  \Qq(\bm{\Theta}; \bm{H}^{\mathrm{old}}) &=\EE_{\hat{p}(\ubm{x}),p(\ubm{s},\ubm{\kappa}| \ubm{x}; \bm{H}^{\mathrm{old}})}\left[  \log\,p(\ubm{x}| \ubm{s},\ubm{\kappa}; \bm{H})\right].
\end{align}

We firstly address the generator learning problem, i.e. $\umax{\bm{\Theta}} \Qq(\bm{\Theta}; \bm{H}^{\mathrm{old}})$. This is boiled down to maximize the cost function of neural networks that can be formulated as
\begin{align}\label{eq:obj-q-gen-mix-log}
  &\Qq(\bm{\Theta}; \bm{H}^{\mathrm{old}}) \nonumber \\
  = &\frac{1}{R}\sum_{r=1}^{R}\sum_{\ubmr{s}{r}}\sum_{\ubmr{\kappa}{r}}{p(\ubmr{s}{r}, \ubmr{\kappa}{r}| \ubmr{x}{r}; \bm{H}^{\mathrm{old}})} \sum_{t=1}^{{T}^{r}}\log\,p(\bmtr{x}{t}{r} | \smtr{s}{t}{r}, \smtr{\kappa}{t}{r}; \bm{H}) \nonumber \\
  =& \frac{1}{R}\sum_{r=1}^{R} \sum_{t=1}^{{T}^{r}} \sum_{\smtr{s}{t}{r}=1}^{|\Ss|}  \sum_{\smtr{\kappa}{t}{r}=1}^{K}p(\smtr{s}{t}{r}| \ubmr{x}{r}; \bm{H}^{\mathrm{old}})p(\smtr{\kappa}{t}{r}|\smtr{s}{t}{r}, \ubmr{x}{r}; \bm{H}^{\mathrm{old}})  \log\, p(\bmtr{x}{t}{r} | \smtr{s}{t}{r}, \smtr{\kappa}{t}{r}; \bm{H}), 
\end{align}
where $T^r$ is the length of the $r$-th sequential data. In \eqref{eq:obj-q-gen-mix-log}, the state posterior $p(s_t| \ubm{x}, \bm{H}^{\mathrm{old}})$ is computed by forward-backward algorithm. The posterior of $\kappa$ is
\begin{align}\label{eq:kappa-posterior}
  p(\kappa| s, \ubm{x}; \bm{H}^{\mathrm{old}})
  &=  \frac{p(\kappa, \ubm{x}| s; \bm{H}^{\mathrm{old}})}{p(\ubm{x}| s,\bm{H}^{\mathrm{old}})} \nonumber \\
  & = \frac{\pi_{s, \kappa}^{\mathrm{old}} p(\bm{x}| s, \kappa, \bm{H}^{\mathrm{old}})}{\sum_{\kappa=1}^{K}  \pi_{s, \kappa}^{\mathrm{old}} p(\bm{x}| s, \kappa,\bm{H}^{\mathrm{old}})},
\end{align}
where the last equation is due to the fact that $\bm{x}_t$ among sequence $\ubm{x}$ only depends on $s_t, \kappa_t$. 

By substituting \eqref{chpt7:eq:change-variable-flow-layer} into \eqref{eq:obj-q-gen-mix-log}, we have cost function for neural networks as
\begin{align}\label{eq:obj-q-gen-mix}
  &\Qq(\bm{\Theta}; \bm{H}^{\mathrm{old}}) \nonumber \\
  =& \frac{1}{R}\hspace{-3pt}\sum_{r=1}^{R}\hspace{-3pt} \sum_{t=1}^{{T}^{r}}\hspace{-3pt} \sum_{\smtr{s}{t}{r}=1}^{|\Ss|} \hspace{-3pt} \sum_{\smtr{\kappa}{t}{r}=1}^{K}p(\smtr{s}{t}{r}| \ubmr{x}{r}; \bm{H}^{\mathrm{old}})p(\smtr{\kappa}{t}{r}|\smtr{s}{t}{r}, \ubmr{x}{r}; \bm{H}^{\mathrm{old}}) \nonumber\\
  &\left[ \log\, p_{\smtr{s}{t}{r}, \smtr{\kappa}{t}{r}}(\bm{f}_{\smtr{s}{t}{r}, \smtr{\kappa}{t}{r}}(\bmtr{x}{t}{r})) + \sum_{l=1}^{L}\log\,| \det (\nabla{\bm{f}_{s,\kappa}^{[l]}})|\right].
\end{align}
The generators of GenHMM simply use standard Gaussian distribution for latent variables $\bm{z} \sim p_{s,\kappa}(\bm{z})$. Since training dataset can be too large to do whole-dataset iterations, batch-size stochastic gradient decent can be used to maximize $\Qq(\bm{\Theta; \bm{H}^{\mathrm{old}}})$ w.r.t. parameters of generators.

In what follows we address the problem $\max_{\bm{\Pi}} \Qq(\bm{\Pi}; \bm{H}^{\mathrm{old}})$ in our generative model learning. The conditional distribution of hidden variable $\kappa$, $\pi_{s, \kappa} = p(\kappa | s; \bm{H})$, is obtained by solving the following problem
\begin{align}\label{opm:pi}
  \pi_{s, \kappa} & = \uargmax{\pi_{s, \kappa}} \Qq(\bm{\Pi}; \bm{H}^{\mathrm{old}}) \\ \nonumber
                  & s.t. \, \sum_{\kappa=1}^{K} \pi_{s, \kappa}= 1, \forall s = 1, 2, \cdots, |\Ss|. 
\end{align}

To solve problem \eqref{opm:pi}, we formulate its Lagrange function as
\begin{equation}
  \Ff = \Qq(\bm{\Pi}; \bm{H}^{\mathrm{old}}) + \sum_{s=1}^{|\Ss|} \lambda_s\left( 1-  \sum_{\kappa=1}^{K}\pi_{s, \kappa}  \right).
\end{equation}
Solving $\pd{\Ff}{\pi_{s, \kappa}} = 0$ gives
\begin{equation}
  \pi_{s,\kappa} = \frac{1}{\lambda_s}\sum_{r=1}^{R} \sum_{t=1}^{{T}^{r}} p(\smtr{s}{t}{r}=s, \smtr{\kappa}{t}{r} =\kappa| \ubmr{x}{r}; \bm{H}^{\mathrm{old}}).
\end{equation}
With condition $\sum_{\kappa=1}^{K} \pi_{s, \kappa}= 1, \forall s = 1, 2, \cdots, |\Ss|$, we have
\begin{equation}
  \lambda_s = \sum_{\kappa=1}^{K}\sum_{r=1}^{R} \sum_{t=1}^{{T}^{r}} p(\smtr{s}{t}{r}=s, \smtr{\kappa}{t}{r} =\kappa | \ubmr{x}{r}; \bm{H}^{\mathrm{old}}).
\end{equation}
Then the solution to \eqref{opm:pi} is
\begin{equation}\label{eq:mix-latent-parameter-solution}
  \pi_{s, \kappa} = \frac{\sum_{r=1}^{R} \sum_{t=1}^{{T}^{r}} p(\smtr{s}{t}{r} =s, \smtr{\kappa}{t}{r}=\kappa | \ubmr{x}{r}; \bm{H}^{\mathrm{old}}) }{\sum_{k =1}^{K}\sum_{r=1}^{R} \sum_{t=1}^{{T}^{r}} p(\smtr{s}{t}{r} =s, \smtr{\kappa}{t}{r}=k | \ubmr{x}{r}; \bm{H}^{\mathrm{old}}) },
\end{equation}
where
\begin{equation}
  p(s, \kappa | \ubm{x}; \bm{H}^{\mathrm{old}}) = p(s| \ubm{x}; \bm{H}^{\mathrm{old}}) p(\kappa | s, \ubm{x}; \bm{H}^{\mathrm{old}}).
\end{equation}
Here $p(s| \ubm{x}; \bm{H}^{\mathrm{old}})$ can be computed by forward-backward algorithm, while $p(\kappa | s, \ubm{x}; \bm{H}^{\mathrm{old}})$ is given by \eqref{eq:kappa-posterior}.


With the generative model learning obtained, it remains to solve the initial distribution update and transition matrix update of HMM in GenHMM, i.e. the problem \eqref{eq:init-distribution-update} and \eqref{eq:transition-update}. These two problems are basically two constrained optimization problems. The solutions to them are available in literature \cite{Bishop:2006:PRM:1162264}. But to keep learning algorithm for GenHMM complete, we give the update rules for $\bm{q}$ and $\bm{A}$ as follows.

\subsubsection{Initial Probability Update}
The problem in \eqref{eq:init-distribution-update} can be reformulated as
\begin{align}
  &\Qq(\bm{q}; \bm{H}^{\mathrm{old}}) \nonumber \\
  =&\frac{1}{R} \sum_{r=1}^{R}\sum_{\ubmr{s}{r}} {p(\ubmr{s}{r}| \ubmr{x}{r}; \bm{H}^{\mathrm{old}})} \log\,p(\smtr{s}{1}{r};\bm{H}) \nonumber \\
  = & \frac{1}{R}\sum_{r=1}^{R}\sum_{\smtr{s}{1}{r}=1}^{|\Ss|}\sum_{\smtr{s}{2}{r}=1}^{|\Ss|}\cdots \sum_{\smtr{s}{T^{r}}{r}}^{{|\Ss|}} {p(\smtr{s}{1}{r}, \smtr{s}{2}{r}, \cdots, \smtr{s}{T^{r}}{r}| \ubmr{x}{r}; \bm{H}^{\mathrm{old}})} \log\,p(\smtr{s}{1}{r}) \nonumber\\
  =& \frac{1}{R}\sum_{r=1}^{R}\sum_{\smtr{s}{1}{r}=1}^{|\Ss|}{p(\smtr{s}{1}{r}| \ubmr{x}{r}; \bm{H}^{\mathrm{old}})} \log\,p(\smtr{s}{1}{r};\bm{H}).
\end{align}

$p(\smtr{s}{1}{r};\bm{H})$ is the probability of initial state of GenHMM for $r$-th sequential sample. Actually $q_i = p({s}_{1} =i;\bm{H}) $, $i= 1, 2, \cdots, |\Ss|$. Solution to the problem
\begin{align}
  \bm{q} \hspace{-0.1cm} = \hspace{-0.1cm} \uargmax{\bm{q}} \Qq(\bm{q}; \bm{H}^{\mathrm{old}}),\; \mathrm{s.t.} \sum_{i=1}^{ |\Ss| }q_i = 1, q_i \geq 0, \forall i.
\end{align}
is
\begin{equation}\label{eq:update-initial-state-prob}
  q_i = \frac{1}{R} \sum_{r=1}^{R} p(\smtr{s}{1}{r}=i | \ubmr{x}{r}; \bm{H}^{\mathrm{old}}), \forall\; i = 1, 2, \cdots, |\Ss|.
\end{equation}

\subsubsection{Transition Probability Update}
The problem \eqref{eq:transition-update} can be reformulated as
\begin{align}
  &\Qq(\bm{A}; \bm{H}^{\mathrm{old}})\nonumber \\
  % &= \sum_{r=1}^{R} \EE_{p(\ubmr{s}{r}| \ubmr{x}{r}; \bm{H}^{\mathrm{old}})} \left[\log\,\sum_{t=1}^{T^{(r)}-1}p(\smtr{s}{t+1}{r}|\smtr{s}{t}{r}; {A})\right] \nonumber\\
  &= \sum_{r=1}^{R} \sum_{\ubmr{s}{r}}{p(\ubmr{s}{r}| \ubmr{x}{r}; \bm{H}^{\mathrm{old}})} \sum_{t=1}^{T^{r}-1}\log\,p(\smtr{s}{t+1}{r}|\smtr{s}{t}{r}; \bm{H}) \nonumber \\
  &= \sum_{r=1}^{R} \hspace{-0.1cm}\sum_{t=1}^{T^{r}-1}\hspace{-0.1cm} \sum_{\smtr{s}{t}{r}=1}^{|\Ss|}\hspace{-0.05cm}\sum_{\smtr{s}{t+1}{r}=1}^{|\Ss|}\hspace{-0.2cm}{p(\smtr{s}{t}{r}, \smtr{s}{t+1}{r}| \ubmr{x}{r};\hspace{-0.05cm} \bm{H}^{\mathrm{old}})} \log\,p(\smtr{s}{t+1}{r}|\smtr{s}{t}{r}; \bm{H}).
\end{align}

Since $\bm{A}_{i, j}  = p(\smtr{s}{t+1}{r}=j|\smtr{s}{t}{r}=i; \bm{H})$ is the element of transition matrix $\bm{A}$, the solution to the problem
\begin{align}\label{eq:update-transition-prob}
  \bm{A} = &\uargmax{\bm{A}} \Qq(\bm{A}; \bm{H}^{\mathrm{old}}) \nonumber \\
  \mathrm{s.t.} &\hspace{0.2cm} \bm{A} \cdot \bm{1} = \bm{1}, \bm{A}_{i,j} \geq 0 \,\, \forall i,j,
\end{align}
is
\begin{equation}
  \bm{A}_{i,j} = \frac{\bar{\xi}_{i,j}}{\sum_{k = 1}^{|\Ss|} \bar{\xi}_{i,k}},
\end{equation}
where
\begin{equation}\label{eq:update-transition-solt}
  \bar{\xi}_{i,j} = \sum_{r= 1}^{R} \sum_{t= 1}^{T^{r}-1}{p(\smtr{s}{t}{r}=i, \smtr{s}{t+1}{r}=j| \ubmr{x}{r}; \bm{H}^{\mathrm{old}})}.
\end{equation}
\subsection{On Convergence of GenHMM}
In pursuit of representing a dataset by GenHMM,  we are interested if the learning solution discussed in subsection~\ref{subsec:optmGenHMM} would converge. The properties on GenHMM's convergence are analyzed as follows.

\begin{proposition}\label{proposition1}
  Assume that parameter $\bm{\Theta} = \left\{ \bm{\theta}_{s,\kappa}| s\in \Ss, \kappa=1, 2, \cdots, K \right\}$ is in a compact set,  $\bm{f}_{s,\kappa}$ and  ${\nabla\bm{f}_{s,\kappa}}$ are continuous w.r.t. ${\bm\theta}_{s,\kappa}$ in GenHMM. Then GenHMM converges.
\end{proposition}

\begin{proof}
  We begin with the comparison of loglikelihood evaluated under $\bm{H}^{\mathrm{new}}$ and $\bm{H}^{\mathrm{old}}$. The loglikelihood of dataset given by $\hat{p}(\ubm{x})$ can be reformulated as
  \begin{align*}
    &\EE_{\hat{p}(\ubm{x})}\left[ \log\,p(\ubm{x};\bm{H}^{\mathrm{new}}) \right] \nonumber \\
    =& \EE_{\hat{p}(\ubm{x}),p(\ubm{s},\ubm{\kappa}| \ubm{x}; \bm{H}^{\mathrm{old}})}\left[ \log\,\frac{p(\ubm{x}, \ubm{s}, \ubm{\kappa}; \bm{H}^{\mathrm{new}})}{p(\ubm{s}, \ubm{\kappa}|\ubm{x}; \bm{H}^{\mathrm{old}})}\right]  \nonumber \\
    & +\EE_{\hat{p}(\ubm{x})}\left[ KL(p(\ubm{s}, \ubm{\kappa}|\ubm{x}; \bm{H}^{\mathrm{old}})\|p(\ubm{s}, \ubm{\kappa}|\ubm{x}; \bm{H}^{\mathrm{new}})) \right],
  \end{align*}
  where the first term on the right hand side of the above inequality can be further written as
  \begin{align*}
    &\EE_{\hat{p}(\ubm{x}),p(\ubm{s},\ubm{\kappa}| \ubm{x}; \bm{H}^{\mathrm{old}})}\left[ \log\,\frac{p(\ubm{x}, \ubm{s}, \ubm{\kappa}; \bm{H}^{\mathrm{new}})}{p(\ubm{s}, \ubm{\kappa}|\ubm{x}; \bm{H}^{\mathrm{old}})}\right] \nonumber \\
    = &\Qq(\bm{H}^{\mathrm{new}}; \bm{H}^{\mathrm{old}}) + \EE_{\hat{p}(\ubm{x}),p(\ubm{s},\ubm{\kappa}| \ubm{x}; \bm{H}^{\mathrm{old}})}\left[p(\ubm{s}, \ubm{\kappa}|\ubm{x}; \bm{H}^{\mathrm{old}})\right].
  \end{align*}
  According to subsection~\ref{subsec:optmGenHMM}, the optimization problems give
  \begin{align*}
    \Qq(\bm{q}^{\mathrm{new}}; \bm{H}^{\mathrm{old}}) &\geq \Qq(\bm{q}^{\mathrm{old}}; \bm{H}^{\mathrm{old}}),\nonumber \\
    \Qq(\bm{A}^{\mathrm{new}}; \bm{H}^{\mathrm{old}}) &\geq \Qq(\bm{A}^{\mathrm{old}}; \bm{H}^{\mathrm{old}}),\nonumber \\
    \Qq(\bm{\Pi}^{\mathrm{new}}; \bm{H}^{\mathrm{old}}) &\geq \Qq(\bm{\Pi}^{\mathrm{old}}; \bm{H}^{\mathrm{old}}), \nonumber \\
    \Qq(\bm{\Theta}^{\mathrm{new}}; \bm{H}^{\mathrm{old}}) &\geq \Qq(\bm{\Theta}^{\mathrm{old}}; \bm{H}^{\mathrm{old}}).
  \end{align*}
  % For the learning w.r.t. neural network parameter set $\bm{\Theta}$, as long as the lose function does not decrease during EM iterations, we would have
  Since
  \begin{align*}
    &\Qq(\bm{H}^{\mathrm{new}}; \bm{H}^{\mathrm{old}}) \nonumber \\
      = &\Qq(\bm{q}^{\mathrm{new}}; \bm{H}^{\mathrm{old}}) + \Qq(\bm{A}^{\mathrm{new}}; \bm{H}^{\mathrm{old}}) + \Qq(\bm{\Pi}^{\mathrm{new}}; \bm{H}^{\mathrm{old}})
                                                          + \Qq(\bm{\Theta}^{\mathrm{new}}; \bm{H}^{\mathrm{old}}),
  \end{align*}
  it gives
  \begin{equation*}
    \Qq(\bm{H}^{\mathrm{new}}; \bm{H}^{\mathrm{old}}) \geq \Qq(\bm{H}^{\mathrm{old}}; \bm{H}^{\mathrm{old}}).
  \end{equation*}
  With the above inequality, and the fact that $\EE_{\hat{p}(\ubm{x}),p(\ubm{s},\ubm{\kappa}| \ubm{x}; \bm{H}^{\mathrm{old}})}\left[p(\ubm{s}, \ubm{\kappa}|\ubm{x}; \bm{H}^{\mathrm{old}})\right]$ is independent of $\bm{H}^{\mathrm{new}}$, we have the inequality 
  \begin{align*}
    &\EE_{\hat{p}(\ubm{x}),p(\ubm{s},\ubm{\kappa}| \ubm{x}; \bm{H}^{\mathrm{old}})}\left[ \log\,\frac{p(\ubm{x}, \ubm{s}, \ubm{\kappa}; \bm{H}^{\mathrm{new}})}{p(\ubm{s}, \ubm{\kappa}|\ubm{x}; \bm{H}^{\mathrm{old}})}\right] 
    \geq \EE_{\hat{p}(\ubm{x}),p(\ubm{s},\ubm{\kappa}| \ubm{x}; \bm{H}^{\mathrm{old}})}\left[ \log\,\frac{p(\ubm{x}, \ubm{s}, \ubm{\kappa}; \bm{H}^{\mathrm{old}})}{p(\ubm{s}, \ubm{\kappa}|\ubm{x}; \bm{H}^{\mathrm{old}})}\right].
  \end{align*}
  Due to $KL(p(\ubm{s}, \ubm{\kappa}|\ubm{x};
  \bm{H}^{\mathrm{old}})\|p(\ubm{s}, \ubm{\kappa}|\ubm{x};
  \bm{H}^{\mathrm{old}}))=0$, we have
  \begin{align*}
    &\EE_{\hat{p}(\ubm{x})}\left[ \log\,p(\ubm{x};\bm{H}^{\mathrm{new}}) \right] \nonumber \\
    \geq & \EE_{\hat{p}(\ubm{x}),p(\ubm{s},\ubm{\kappa}| \ubm{x};
           \bm{H}^{\mathrm{old}})}\left[ \log\,\frac{p(\ubm{x},
           \ubm{s}, \ubm{\kappa}; \bm{H}^{\mathrm{old}})}{p(\ubm{s},
           \ubm{\kappa}|\ubm{x}; \bm{H}^{\mathrm{old}})}\right] \nonumber \\
           &+ \EE_{\hat{p}(\ubm{x})}\left[ KL(p(\ubm{s}, \ubm{\kappa}|\ubm{x}; \bm{H}^{\mathrm{old}})\|p(\ubm{s}, \ubm{\kappa}|\ubm{x}; \bm{H}^{\mathrm{old}})) \right]
      \nonumber \\
    = & \EE_{\hat{p}(\ubm{x})}\left[ \log\,p(\ubm{x};\bm{H}^{\mathrm{old}}) \right].
  \end{align*}
  Since $\bm{f}_{s,\kappa}$ and  ${\nabla\bm{f}_{s,\kappa}}$ are continuous w.r.t. ${\bm\theta}_{s,\kappa}$ in GenHMM, $\EE_{\hat{p}(\ubm{x})}\left[ \log\,p(\ubm{x};\bm{H}) \right]$ is bounded. The above inequality shows $\EE_{\hat{p}(\ubm{x})}\left[ \log\,p(\ubm{x};\bm{H}) \right]$ is non-decreasing in learning of GenHMM. Therefore, GenHMM will converge.
  
  % Therefore
  % \begin{align*}
  %   KL(\hat{p}(\ubm{x})\|p(\ubm{x};\bm{H}^{\mathrm{new}})) = &\EE_{\hat{p}(\ubm{x})}\left[ \log\,\frac{\hat{p}(\ubm{x})}{p(\ubm{x};\bm{H}^{\mathrm{new}})} \right] \nonumber \\
  %   \leq &\EE_{\hat{p}(\ubm{x})}\left[ \log\,\frac{\hat{p}(\ubm{x})}{p(\ubm{x};\bm{H}^{\mathrm{old}})} \right] \\
  %   = & KL(\hat{p}(\ubm{x})\|p(\ubm{x};\bm{H}^{\mathrm{old}})).
  % \end{align*}
  % Since KL divergence is non-negative and thus lower bounded, GenHMM will converge.
\end{proof}

\subsection{Algorithm of GenHMM}
\begin{algorithm}[H]
  \caption{Learning of GenHMM}\label{algo:genhmm}
  \begin{algorithmic}[1]
    \STATE {\bfseries Input:}{
      Empirical distribution $\hat{p}(\bm{x})$ of dataset}\\
    \STATE Initializing $\bm{H}^{\mathrm{old}}, \bm{H} \in \Hh$ gives: \\
    $\bm{H}^{\mathrm{old}} = \{\Ss, \bm{q}^{\mathrm{old}}, A^{\mathrm{old}}, p(\bm{x}|s; \bm{\Phi}_{s}^{\mathrm{old}})\}$, \\
    $\bm{H} = \{\Ss, \bm{q}, A, p(\bm{x}|s; \bm{\Phi}_{s})\}$, \\
    in which generators $\left\{\bm{g}_{s,\kappa}|s\in \Ss, \kappa=1,
      2, \cdots, K \right\}$ are all initialized randomly.
    \STATE $\bm{H}^{\mathrm{old}} \gets \bm{H}$
    \STATE Set learning rate $\eta$, neural network optimization batches $N$ per EM step
    \FOR { $\bm{H}$ not converge}
    \FOR {epoch $n < N$}
    \STATE Sample a batch of data $\left\{ \ubmr{x}{r} \right\}_{r=1}^{R_b}$ from dataset $\hat{p}(\ubm{x})$ with batch size $R_b$

    \STATE Compute posterior $p(\smtr{s}{t}{r}, \smtr{\kappa}{t}{r}| \ubmr{x}{r}; \bm{H}^{\mathrm{old}})$  
    \STATE Formulate loss ${\Qq}\left({\bm{\Theta}}, {\bm{H}}^{\mathrm{old}}\right)$ in \eqref{eq:obj-q-gen-mix}

    \STATE $\partial{\bm{\Theta}} \gets  \nabla_{\bm{\Theta}} {\Qq}\left({\bm{\Theta}},{\bm{H}}^{\mathrm{old}}\right)$
    \STATE $\bm{\Theta} \gets \bm{\Theta} + \eta \cdot \partial{\bm{\Theta}}$
    \ENDFOR
    \STATE $\bm{q} \gets \uargmax{\bm{q}}\, \Qq(\bm{q}; \bm{H}^{\mathrm{old}})$ by \eqref{eq:update-initial-state-prob}
    \STATE $\bm{A} \gets \uargmax{\bm{A}}\Qq(\bm{A}; \bm{H}^{\mathrm{old}})$ by \eqref{eq:update-transition-solt}
    \STATE $\bm{\Pi} \gets \uargmax{\bm{\Pi}}\Qq(\bm{\Phi}; \bm{H}^{\mathrm{old}})$ by \eqref{eq:mix-latent-parameter-solution}
    \STATE $\bm{H}^{\mathrm{old}} \gets \bm{H}$
    \ENDFOR
  \end{algorithmic}
\end{algorithm}

To summarize the learning solution in section~\ref{subsec:optmGenHMM}, we wrap our algorithm into pseudocode as shown in Algorithm~\ref{algo:genhmm}. We use Adam \cite{DBLP:journals/corr/KingmaB14} optimizer for optimization w.r.t. the parameters of generators in GenHMM. As shown from line $6$ to $10$ in Algorithm~\ref{algo:genhmm}, the batch-size stochastic gradient decent can be naturally embedded into the learning algorithm of GenHMM.

As described by the pseudocode in Algorithm~\ref{algo:genhmm}, the learning of GenHMM is divided into optimizations w.r.t. to generators' parameters $\bm{\Theta}$, initial probability $\bm{q}$ of hidden state, transition matrix $\bm{A}$, and generator mixture weights $\bm{\Pi}$. Different from the optimization w.r.t. to $\bm{q}$, $\bm{A}$ and $\bm{\Pi}$, which have optimal solutions, generator learning usually cannot give optimal solution to problem $\max_{\bm{\Theta}} \Qq(\bm{\Theta}; \bm{H}^{\mathrm{old}})$. In fact, given that no optimal $\bm{\Theta}$ is obtained, learning of GenHMM can still converge as long as quantity $\Qq(\bm{\Theta}; \bm{H})$ are improving in iterations in Algorithm~\ref{algo:genhmm}, where the inequalities in Proposition~\ref{proposition1} still hold. Therefore optimal $\bm{\Theta}$ in each iteration is not required for convergence of GenHMM as long as the loss in $\eqref{eq:obj-q-gen-mix}$ is getting improved.
\begin{remark}
  \label{chpt7:rmk:generative-discriminative}
  {The above discussed maximum likelihood training is also known as generative training, which models the signal generating process with the defined generative model. An alternative training principle is discriminative training where a conditional distribution is modeled directly and optimized as the objective. Since generative models characterize a joint distribution, it is usually possible to train a generative model via discriminative training, with mildly alternating the objective function. On the contrast, it is usually not likely to train a discriminative model via generative training, since the direct conditional probability modeling of the discriminative model omits modeling distribution of the observable signal that is highly structured. These two training principles are possible to be used together at different training phases. For instance, apply the generative training firstly, and then a discriminative training phase for the generative-trained model which acts as a parameter fine-tuning phase \cite{hinton2012deepSpeech}. See detailed insightful discussion about these two principles by \cite{lasserre2006principled}.
  }
\end{remark}

\section{Application to Speech Recognition}\label{chpt:7:sec:app-speech-recog}
To show the validity of our model, we implement our model in PyTorch and test it with speech sequential data. We first discuss the experimental setups and then show the experimental results. 

\subsection{Experimental Setup}
The dataset used for sequential data modeling and classification is TIMIT where the speech signal is sampled at $16$kHz.
The TIMIT dataset consists of $5300$ phoneme-labeled speech utterances which are partitioned into two sets: {a train set consists of $4620$ utterance, and a test set consists of $1680$ utterances.} There are totally $61$ different types of phones in TIMIT.
We performed experiments in two cases: i) full $61$-phoneme classification case; ii) $39$-phonme classification case, where $61$ phonemes are folded onto $39$ phonemes as described in \cite{Perdigao11}.

For extraction of feature vectors, we use $25$ms frame length and $10$ms frame shift to convert sound track into standard Mel-frequency cepstral coefficients (MFCCs) features. Experiments using the deltas and delta-deltas of the features are also carried out.


Our experiments are performed for: i) standard classification tasks (Table~\ref{tab:acc-classification39f_a}, \ref{tab:acc-classification39f_b}, \ref{tab:acc-classification13f_a}, \ref{tab:acc-classification13f_b}), ii) classification under noise perturbation (table~\ref{tab:acc-classification39f_noise_snr}, \ref{tab:acc-classification39f_noise_type}). The criterion used to report the results includes accuracy, precision and F1 scores.
In all experiments, generators $\left\{\bm{g}_{s,\kappa}|s\in \Ss, \kappa=1, 2, \cdots, K \right\}$ of GenHMM are implemented as flow models. Specifically, our generator structure follows that of a RealNVP described in \cite{2016arXiv160508803D}.
As discussed, the coupling layer shown in \eqref{chpt6:eq-gl-coupling} maps a part of its input signal identically.
The implementation is such that layer $l+1$ would alternate the input signal order of layer $l$ such that no signal remains the same after two consecutive coupling layers.
We term such a pair of consecutive coupling layers as a \textit{flow block}.
In our experiments, each generator $\bm{g}_{s,\kappa}$ consists of four \textit{flow blocks}.
The density of samples in the latent space is defined as Normal,
i.e. $p_{s,\kappa}(\bm{z})$ is the density function of standard
Gaussian. The configuration for each generator is shown as Table~\ref{table:generator-setting}.

\begin{table}
  \caption{Configuration of generators of GenHMM in Experiments}\label{table:generator-setting}
  \centering
  \begin{tabular}{lc}
    \toprule
    \begin{tabular}[x]{@{}c@{}} Latent distribution $p_{s,\kappa}(\bm{z})$ \\ $s\in \Ss, \kappa=1, 2, \cdots, K $ \end{tabular} & Standard Gaussian \\
    \hline
    Number of flow blocks & $4$ \\
    \hline
    Non-linear mapping $\bm{m}_a$, $\bm{m}_b$ & \begin{tabular}[x]{@{}c@{}} Multiple layer perception \\ $3$ layers and with hidden dimension $24$ \end{tabular}\\
    \bottomrule
  \end{tabular}
\end{table}



For each GenHMM, the number of states is adapted to the training dataset.
The exact number of states is decided by computing the average length of MFCC frames per phone in training dataset, and clipping the average length into $\left\{ 3,4,5 \right\}$.
Transition matrix $\bm{A}$ is initialized as upper triangular matrix for GenHMM.


\begin{table}
  \caption{Test accuracy table for $39$ dimensional features and folded $39$ phonemes.}\label{tab:acc-classification39f_a}
  \centering  
  \begin{tabular}{llccc}
    \toprule
    {Model} & Criterion &  K=1 &  K=3 &  K=5  \\  \midrule
    \multirow{3}{*}{GMM-HMM}
            & Accuracy    & 62.3 &  68.0 &  68.7  \\
            & {Precision} & 67.9 &  72.6 &  73.0  \\
            & {F1}        & 63.7 &  69.1 &  69.7 \\
    \midrule
    \multirow{3}{*}{GenHMM}
            & Accuracy    & 76.7   & \textbf{77.7} &  {77.7} \\ 
            & {Precision} & 76.9   & \textbf{78.1} &  78.0 \\
            & {F1}        & 76.1   & \textbf{77.1} &  77.0\\
    \bottomrule                                                                  
  \end{tabular}
\end{table}

\begin{table}
  \caption{Test accuracy table for $39$ dimensional features and $61$ phonemes.}\label{tab:acc-classification39f_b}
  \centering  
  \begin{tabular}{llccc} \toprule
    {Model} & Criterion & K=1 &  K=3 &  K=5
    \\ \midrule
    \multirow{4}{*}{GMM-HMM}
            & Accuracy    & 53.6 &  59.6 & 61.9  \\
            & {Precision} & 59.1 &  63.9 & 65.7 \\
            & {F1}        & 54.7 &  60.5 & 62.7\\
    \midrule
    \multirow{3}{*}{GenHMM}
            & Accuracy    & 69.5 & 70.6 & \textbf{70.7}   \\
            & {Precision} & 69.2 & 70.5 & \textbf{71.0}   \\
            & {F1}        & 68.6 & 69.6 & \textbf{69.6}   \\
    \bottomrule
  \end{tabular}
  \vspace{0.5cm}
\end{table}


\subsection{Numerical Results}

We firstly show the phoneme classification using 39 dimensional MFCC features (MFCC coefficients, deltas, and delta-deltas), to validate one possible usage of our proposed model. Since generative training is carried out in our experiments, GMM-HMM is trained and tested as a reference model in our experiments. Training and testing of GMM-HMM is in the same condition as GenHMMs are trained and tested. Dataset usage for GenMM and GMM-HMM is the same, and number of states for GMM-HMM is the same as that for GenHMM in modeling each phoneme. Apart from setting the reference model, we also run the experiment comparisons with different total number of mixture components.

Table \ref{tab:acc-classification39f_a} and \ref{tab:acc-classification39f_b} shows the results for this experiments, in which we test both the folded $39$-phoneme classification case (the conventional way) in Table~\ref{tab:acc-classification39f_a} and the $61$-phoneme classification case in Table~\ref{tab:acc-classification39f_b}. As shown in both $61$-phoneme and $39$-phoneme cases, GenHMM gets significant higher accuracy than GMM-HMM for the same number of mixture components. The comparisons with regarding to precision and F1 scores show similar trends and also demonstrate significant improvement of GenHMM's performance. As our expectation, GenHMM has better modeling capacity of sequential data since we bring in the neural network based generators into GenHMM, which should be able to represent complex relationship between states of HMM and sequential data. Apart from the gain of using neural network based generative models, there are also increases of accuracy, precision and F1 scores as the number of mixture components in GenHMM is increased from $K=1$ to $K=5$. The sequential dependency of data is modeled by HMM itself, while each state of HMM can have better representation using a mixture probabilistic model if data represented by the state is multi-mode. Comparing the results in $39$-phoneme and $61$-phoneme cases, GenHMM gets higher accuracy for $39$-phoneme classification than it does for $61$-phoneme classification. The total training dataset size remains the same as $61$ phonemes are folded into $39$ phonemes. There are less training data available per phonemes and more classes to be recognized in the $61$-phoneme case, which makes the task more challenging.

\begin{table}
  \caption{Test accuracy table for $13$ dimensional features and folded $39$ phonemes.}\label{tab:acc-classification13f_a}
  \centering  
  \begin{tabular}{llccc}
    \toprule
    {Model} & Criterion & K=1 &  K=3 &  K=5  \\  \midrule
    \multirow{3}{*}{GMM-HMM}
            & Accuracy & 48.5 &  51.2 &  52.4  \\
            & Precision& 56.2 &  58.3 &  59.5  \\
            & F1       & 50.3 &  53.0 &  54.2  \\
    \midrule
    \multirow{3}{*}{GenHMM}
            & Accuracy & 61.1 &  \textbf{62.1} &  62.1   \\ 
            & Precision& 61.1 &  61.9 &  \textbf{62.1}  \\
            & F1       & 59.7 &  \textbf{60.7} &  60.2  \\

    \bottomrule                                                                  
  \end{tabular}
\end{table}

\begin{table}
  \caption{Test accuracy table for $13$ dimensional features and $61$
    phonemes.}\label{tab:acc-classification13f_b}
  \centering
  \begin{tabular}{llccc}
    \toprule
    {Model} & Criterion &  K=1 &  K=3 &  K=5 \\ \midrule
    \multirow{3}{*}{GMM-HMM}
            & Accuracy & 37.1 &  40.6 & 42.2  \\
            &Precision & 44.6 &  47.4 & 48.8  \\
            & F1       & 38.8 &  42.1 & 43.7 \\
    \midrule
    \multirow{3}{*}{GenHMM}
            & Accuracy & 50.3 & 50.8 & \textbf{52.3}   \\
            & Precision& 49.3 & 50.9 & \textbf{52.1} \\
            &F1        & 47.8 & 48.3 & \textbf{49.3} \\
    \bottomrule
  \end{tabular}
  \vspace{0.5cm}
\end{table}

Similar experiments are carried out by using only the MFCC coefficients as feature input (excluding deltas and delta-deltas). The results are shown in Table~\ref{tab:acc-classification13f_a} and \ref{tab:acc-classification13f_b}. The superior performance of GenHMM remains compared with reference model GMM-HMM, with regarding to accuracy, precision and F1 scores. The gain by using mixture generators is also presented in this set of experiments while the difference between $61$-phoneme and $39$-phoneme cases is similar to the set of experiments in Table~\ref{tab:acc-classification39f_a} and \ref{tab:acc-classification39f_b}.

\begin{table}
  \caption{Test accuracy table of perturbation with white noise ($K=3$, folded $39$ phonomes).}
  \label{tab:acc-classification39f_noise_snr}
  \centering
  \begin{tabular}{llcccc}
    \toprule
    \multirow{2}{*}{Model} & \multirow{2}{*}{Criterion} &
                                                          \multicolumn{4}{c}{White Noise SNR} \\
    % \cline{3-6}
    
                           && {15dB} &  {20dB} &  {25dB} & {30dB}  \\
    \midrule
    \multirow{3}{*}{GMM-HMM}
                           & Accuracy & 36.6 &  44.2 &  50.8 & 57.1
    \\
                           &Precision & 59.2 &  64.2 &  68.4 & 70.6  \\
                           & F1       & 39.9 &  47.7 &  53.9 & 59.9 \\
    \midrule
    \multirow{3}{*}{GenHMM}
                           & Accuracy & 52.4 & 62.0 &  69.7 & \textbf{74.3} \\
                           &Precision & 60.0 &  65.9 &  71.7 & \textbf{74.8}  \\
                           & F1       & 52.5 &  62.0 &  69.3 & \textbf{73.5} \\
    \bottomrule                                                                  
  \end{tabular}
  \vspace{0.1cm}
\end{table}
\begin{table}
  \caption{Test accuracy table of perturbation by different type of noise (SNR=$20$dB, $K=3$, folded $39$ phonomes).}
  \label{tab:acc-classification39f_noise_type}
  
  \centering
  \begin{tabular}{llcccc}
    \toprule
    \multirow{2}{*}{Model} & \multirow{2}{*}{Criterion} &
                                                          \multicolumn{4}{c}{Noise Type} \\
   
    
                           &  &  White &  Pink &  Babble & Volvo  \\
    \midrule
    \multirow{3}{*}{GMM-HMM}
                           & Accuracy & 44.2 &  48.8 &  57.7 & {66.6}
    \\
                           &Precision & 64.2 &  66.1 &  67.0 & {71.9}  \\
                           & F1       & 47.7 &  52.3 &  59.7 & {67.8} \\
    \midrule
    \multirow{3}{*}{GenHMM}
                           & Accuracy & 62.0 &  65.1 &  70.0 & \textbf{75.7} \\
                           &Precision & 65.9 &  67.8 &  70.4 & \textbf{75.9}  \\
                           & F1       & 62.0 &  64.6 &  69.0 & \textbf{75.3} \\
    \bottomrule                                                                  
  \end{tabular}
  \vspace{0.3cm}
\end{table}

Apart from standard classification testing, we also test the robustness of our model to noise perturbations. We train GenHMM with $K=3$ by clean TIMIT training data in the case of folded $39$ phonemes with $39$ dimensional features. The testing dataset is perturbed by either the same type of noise with different signal-to-noise ratio (SNR) as shown in Table~\ref{tab:acc-classification39f_noise_snr}, or different type of noises with the same SNR as shown in Table~\ref{tab:acc-classification39f_noise_type}. The noise data is from NOISEX-92 database. The baseline of these two sets of experiments is the accuracy testing of GenHMM and GMM-HMM on clean testing data in the same experimental condition, where GenHMM has $77.7\%$ and GMM-HMM gets $68.0\%$ as shown in Table~\ref{tab:acc-classification39f_a}. Similar superior performance of GenHMM with regarding to precision and F1 scores is also shown. It is shown in Table~\ref{tab:acc-classification39f_noise_snr} that GMM-HMM's performance degenerates more than GenHMM's performance at the same level of noise perturbation, though the accuracy of both models increases along the increase of SNR. Especially, for SNR=$30$dB, the accuracy of GenHMM drops only about $3\%$ (from $77.7\%$ to $74.3\%$), while GMM-HMM encounters more than $10\%$ decrease (from $68.0\%$ to $57.1\%$) due to the noise perturbation. In Table~\ref{tab:acc-classification39f_noise_type}, the SNR remains constant and GenHMM is tested with perturbation of different noise types. It is shown that GenHMM still remain higher performance scores at different types of noise perturbations than GMM-HMM. Among these four types of noise, white noise shows most significant impact to GenHMM while the impact of volvo noise is negligible.


\section{Application to Sepsis Detection}\label{chpt7:sec:app-sepsis}
In this section, we apply our developed model to the sepsis detection for infants. Sepsis may rapidly develop among newborn babies who are under care in neonatal intensive care units (NICU). The sequential signals in this application are the physiological signals, which are taken as input for
sepsis prediction.

Apart from the default maximum likelihood that does generative training of GenHMM, it is also possible to carry out discriminative training as discussed in Remark~\ref{chpt7:rmk:generative-discriminative}.
This discriminative training is performed by maximizing the conditional probability of the correct class $y\in \Yy$ given its input $\ubm{x}$, where $\Yy$ is the alphabet of class $y$. This leads to
\begin{equation} \label{chpt7:eq:discriminative}
\umax{\left\{ \bm{H}_i, i=1, \cdots \abs{\Yy} \right\}}{}\sum_{(\ubm{x},\bm{y})} \log \frac{p(\ubm{x}\mid\bm{H}_{{y}})p(\bm{H}_{{y}})}{\sum_{y^{\prime}}p(\ubm{x}\mid\bm{H}_{y^{\prime}})p(\bm{H}_{y^{\prime}})}.
\end{equation}
It can be seen that the complexity in discriminative training of GenHMM increases as $\abs{\Yy}$ is large, since the marginal $p(\ubm{x})= \sum_{y^{\prime}}p(\ubm{x}\mid\bm{H}_{y^{\prime}})p(\bm{H}_{y^{\prime}})$ has to be evaluated for each sequential sample. But for this sepsis detection, there are two classes, i.e. $\Yy = \left\{ 1,2 \right\}$ and thus the computation is affordable.
The classes prior probabilities $p(\bm{H}_{i}),~i=1,2$ are infered from the training dataset. We use \textit{dGenHMM} to denote that GenHMM is finely tuned by discriminative training with objective \eqref{chpt7:eq:discriminative}, after default generative training phase.

\subsection{Patient Dataset}
The bedside monitor signals of $48$ premature infants which have been under care at a NICU in Karolinska University hospital have been collected.
The sequential signals used are the Respiratory Frequency (RF), the beat to beat interval (RRi) and the blood oxygen saturation level ($Sp0_2$).
All signals were sampled at $1$Hz and segmented into $20$ minutes time frames.
Each time frame was then labelled based on information retrieved from the Electronic Health Records (EHR).
Similar to heart rate observation system (HeRO) \cite{hicksHeartRateObservation2013, griffinHeartRateCharacteristics2005}, a logistic regession method using RRi, we aim at detecting septic events earlier than clinical suspicion of sepsis.
In our study we use a threshold of $72$h prior to blood sample, to label a time frame as \textit{septic}, according to the practices in \cite{gurMathematicalAlgorithmDetection2014}.
A time frame was retro-actively labeled $1$ if it occurred at most $72$h prior to clinical suspicion.
A time frame was labeled $0$ if it occured during a day when no notes were entered in the infant's EHR.
All time frames not labeled either $0$ or $1$ were discarded.

Our final dataset consists of $22$ patients, among which $13$ males and $9$ females.
The birth weight was $1.61\pm1.10$ kg and the gestational age at birth $30.9\pm6.14$.
Our dataset consisted in $3501$ time frames, among which 1774 with label $0$ and $1727$ with label $1$.
All time frames have a constant size of $T=1200$ samples and are $3$-dimensional.

\subsection{Experimental Setup}
To make the experiments more interesting, we add a set of basedline methods in the comparisons.
Baseline model HeRO is used, which uses the RRi singal for feature extraction and applies a logistic regression to the extracted features. See \cite{honore2020icassp} about detailed feature extraction.
Pulse oximetry predictive score (POPS) \cite{sullivanEarlyPulseOximetry2018} uses mean, standard  deviation,  skewness,  kurtosis  and  min-maxcross correlation between RR-interval and $SpO2$ to compute a risk score. POPS is also a logistic regression method. 
For the logistic regressions, optimal regularization parameter was found with cross-validation and grid search in the set $\{10^{-5}, \cdots, 10^{5}\}$.

We performed binary classification by the maximum likelihood with GMM-HMM, GenHMM and dGenHMM as probabilistic models.
The GMM-HMM hyper-parameters were the number of states and the number of Gaussians per state.
% The number of Gaussians per states was varied between \{2,4,6,8,10,12\}.
For GenHMM and dGenHMM, the hyper-parameters were the number of states, the number of flows per state, chain-length in each flow, the size of the networks in a coupling layer of a flow.
Given our limited input dimension, the size of the networks in the coupling layers was fixed to $3$.
The chain-length in the coupling layer of each flow was varied between $4$ and $8$.
We varied the number of states in HMMs in $\{3,6,9\}$ and show the results in Table~\ref{table:results:HMM}.

We repeated our experiments $3$ times and each time a random $30\%$ of the patients was left out for testing.
This lead to $2361 \pm 353$ time series in the training sets and $1140 \pm 353$ time series in the testing sets.
We used a different set of input features to test our models in different conditions.
HMMs are trained on raw time series and on raw time series with first and second order derivatives.
For logistic regression models, HeRO is trained with $3$-dimensional features, and POPS with $10$-dimensional features.
The results associated with these two sets of features are presented in Table~\ref{table:results:Lin}.
Extreme learning machines (ELM) and support vector machine (SVM) with a gaussian kernel trained on the raw time series data are also added into the comparison.

\subsection{Numerical Results}
The results for the HMMs are presented in Table~\ref{table:results:HMM} and the comparison with other benchmarks are presented in Table~\ref{table:results:Lin}.

\begin{table}[!ht]
\center
\caption{Test accuracy of HMMs}
\setlength{\tabcolsep}{4pt}
\begin{tabular}{cccc}    
\toprule
Number of states 		&  n=3 			    				&  n=6									&  n=9 						\\  
\hline
                                        \multicolumn{4}{c}{Raw sequential input} 		\\
\hline
GMM-HMM				&   0.68 $\pm$ 0.03 		&    \textbf{0.68} $\pm$ 0.03 			&   	\textbf{0.69} $\pm$ 0.03  	\\
GenHMM  				&   0.67 $\pm$ 0.04     	&    0.61 $\pm$ 0.08   		    &   	0.63 $\pm$ 0.08   \\
dGenHMM 				&   \textbf{0.70} $\pm$ 0.10    	&	   0.67 $\pm$ 0.06      	    &  	    0.65 $\pm$ 0.04    \\
   \toprule
                                        \multicolumn{4}{c}{Raw sequential input + $1^{st}$ and $2^{nd}$ order derivatives}\\
\hline
GMM-HMM				&   \textbf{0.75} $\pm$ 0.05 		&    \textbf{0.74} $\pm$ 0.08			&   	\textbf{0.74} $\pm$ 0.05  	\\
GenHMM  				&   0.69 $\pm$ 0.07    	&    0.66 $\pm$ 0.06  		    &   	0.59 $\pm$ 0.08  \\
dGenHMM 				&  0.71 $\pm$ 0.04    		&	   0.72 $\pm$ 0.10     	    &  	   0.67 $\pm$ 0.04   \\
\bottomrule
\end{tabular}

\label{table:results:HMM}
\end{table}

%%%%%%%%%%%%%%%%%%%%
%%%%%%%%%%%%%%%%%%%%
\begin{table}[!ht]
\center
\caption{Comparison of HMM with other models}
\begin{tabular}{*4c}
\toprule
%                                       			& \multicolumn{2}{c}{Features}\\
 Model							&   			    & Performance 	 	\\  
\toprule

\multicolumn{4}{c}{Using feature extraction}    \\ \hline
Logistic Regression	 & & POPS & 0.54$\pm$0.01				  \\
& & HeRO			&   0.57$\pm$0.04\\   \hline
\multicolumn{3}{c}{Using raw sequential input}    \\    \hline  
SVM      & &                       &   0.60 $\pm$ 0.04   			\\
ELM		& &						&   0.60 $\pm$ 0.03   				\\ 
dGenHMM     & &                     &   0.70 $\pm$ 0.10  \\ \hline

\hline
\end{tabular}
\label{table:results:Lin}
\end{table}%\vspace{-20pt}
%%%%%%%%%%%%%%%%%%%%
%%%%%%%%%%%%%%%%%%%%


The HeRO reaches $57\%$ of correct classifications.
It outperforms the POPS algorithms which reaches only $54\%$, which is method with the lowest performance here.
As expected, the logistic classifiers are outperformed by Gaussian kernel SVM, ELM and our HMMs.
SVM and ELM both reach an accuracy of $60\%$ and comparable standard deviations of $4\%$ and $3\%$.
This is lower than dGenHMM which reaches $70\%$ accuracy and outperforms both GenHMM $67\%$, and GMM-HMM $68\%$.
These results are contrasted by the large standard deviation of dGenHMM $10\%$, which is larger than both GenHMM $4\%$ and GMM-HMM $3\%$.
When the number of states increases to $n=6$ and $n=9$, GMM-HMM reaches $68\%$ and $69\%$ which outperforms GenHMM, and dGenHMM.
When the raw sequential input is augmented with $1^{st}$ and $2^{nd}$ order derivatives, GMM-HMM reaches its highest performance at $75\%$ accuracy.
GenHMM and dGenHMM also reach their highest performance with $69\%$ at $n=3$ and $72\%$ at $n=6$.

Improved performance of GenHMM using discriminative training (dGenHMM) was significant for both $1^{st}$ and $2^{nd}$ order derivatives as argument to inputs.
This is encouraging, given that our current discriminative training consists of only one epoch, i.e. one-epoch fine-tuning.
The performance of GMM-HMM is also significantly increased when adding $1^{st}$ and $2^{nd}$ order derivative, reaching higher than dGenHMM.
This phenomenon is due to two aspects: i) model complexity; ii) dataset size. GMM-HMM is a simpler model compared to GenHMM. Besides, the number of samples in dataset is limited, i.e. only $3501$ time frames. These two major factors together explain why the performance of GMM-HMM increase while that of GenHMM decreases as $n$ increase from $3$ to $9$ (a increase of each corresponding model's complexity).
Also discriminative trianing, dGenHMM, does improve the performance. The insufficient training data makes its final accuracy fall behind the much simpler model GMM-HMM.

\section{Summary}
In this chapter, we discussed the representation and learning of a temporal model. With the simplification of Markov assumption and the concept of template, the HMMs, as simplified dynamic Bayesian networks, have a good balance between representation power and computation efficiency. Thus, they have been widely used in disciplines in practice. The modeling expressivity is further enriched by the introduction of normalizing flows. The developed GenHMM is a generic model and allows large freedom of customization, which can be employed for practical applications such as the demonstrated speech recognition and sepsis detection. Due to the presence of hidden variables, the learning of GenHMM is carried out within expectation maximization framework. Adaption is made to accommodate the neural network implementation of normalizing flows and large datasets in parameter learning.

In increasing the representation power of modeling, GenHMM relies on enriching the flexibility of conditional probability models. Alternative options lie with the template model design. In fact, template is the core concept for more general object-relational modeling of probabilistic graphical models. Generally speaking, any object-relation can be used to define a skeleton that functions as a template. Then ground probabilistic graphical models can be formed by assembling the instances from skeletons or templates. The results models can be rich and representative for general object-relation modeling. The sequential instances concatenation shown in the chapter is one way of the general modeling process. It is suggested to bear the learning complexity in mind when making use of flexible representation of template models. It may be the case that the best representation brings prohibitive inference or learning complexity. Therefore, the modeling accuracy and inference/learning complexity should be jointly considered when designing models for practical applications.

\section{Literature Work}
Probabilistic models of temporal processes are challenging topics in pattern recognition and machine learning, which can be generalized to a wide range of problems of sequential dependency beyond temporal signals.
The various applications cover but not limited to reinforcement learning \cite{ding2018reinforcementhmm,levine2018reinforcementReview}, natural language modeling \cite{khan2016survey,Hariyanti_2019}, biological sequence analysis such as proteins \cite{ASHWIN20172} and DNA \cite{ren2015dna}, etc.
The study on temporal process modeling goes back many years.
Hidden Markov models (HMMs) were discussed as early as in $1980$s by \cite{rabiner1986intro_hmm}, and further by \cite{rabiner1989tutorialhmm}.
The generalization of HMMs into probabilistic graphical models, dynamic Bayesian networks, was proposed around the same period in \cite{dean1989reasoning}.

The classic way is to use a Gaussian mixture model (GMM) per state of HMM \cite{juang1986maximum}, where GMMs are used to connect states of HMM to sequential data input. GMM based HMM (GMM-HMM) has become a standard model for sequential data modeling, and been employed widely for practical applications, especially in speech recognition \cite{gales2008application,chatterjee2011auditory}.

To represent data in nonlinear manifolds, research attempting at training HMM with neural networks has been carried out to boost the modeling capacity of HMM. A successful work of this track has brought deep neural network (DNN) that is defined by restrictive Boltzmann machines (RBMs) \cite{Hinton2012} into HMM based models \cite{hinton2012deepSpeech,li2013hybrid,Miao2013ImprovingLC}. RBM based HMM is trained with a hierarchical scheme consisting of multiple steps of unsupervised learning, formatting of a classification network, and then supervised learning. The hierarchical procedure comes from the empirical expertise in this domain. The hierarchical learning scheme of RBM/DNN based HMM consists of: i) RBMs are trained one after the other in an unsupervised fashion and are stacked together as one deep neural network model, ii) then a final softmax layer is added to the stack of RBMs to represent the probability of an HMM state given a data input, iii) a discriminative training is performed for the final tuning of the model at the final stage.

With the prevalence of deep learning, further works on neural network based temporal models are present in literature. A representative track is the hybrid method of temporal neural network models and HMM. In \cite{liu2019lstmHmmHyb,buys2018bridging,vik2016rnnHmm}, a long short-term memory (LSTM) model/recurrent neural network (RNN) is combined with HMM as hybrid. A hierarchical training is carried out by: i) training an HMM first, ii) then doing modified training of LSTM using trained HMM. This hierarchical training procedure is motivated by the intuition of using LSTM or RNN to fill in the gap where HMM can not learn.
Another successful exploration is motivated by hierarchical HMM \cite{fine1998hierarchicalHMM}, a variant of HMM that tries to model complex multi-scale structures of hidden variables in sequential signal modeling. Work hierarchical multiscale RNN \cite{chung2016hmRNN} and variational temporal abstraction \cite{kim2019variational} use neural networks to infer the multi-scale structures of hidden variables, where applications of language models and navigations are demonstrated successfully.


%%% Local Variables:
%%% mode: latex
%%% TeX-master: "../../main"
%%% End:

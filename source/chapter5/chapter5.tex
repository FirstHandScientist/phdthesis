\chapter{Learning with inference}


\section{learning Undirected graphical models/ MRF}
move the MRF learning by using RENN here

I should read lecture note 7 of 10-708 again when writing this seciton.

\section{Amortized/Neural Variational Learning and Inference of partial observed MRF}
1. TRW as upper bound to partition function

2. Mean field or negative TRW as lower bound to partition function

combining above together, we can obtain two different lower bound of likelihood. Consider if worthy a paper.

\begin{itemize}
\item The log-likelihood of partial observed MRF is non-convex in general ( log-sum-exp is convex, but the difference of two log-sum-exp functions might not be). This combination convert the original non-convex learning into convex optimization with regarding to MRF parameter? should be, but need a confirmation.
\item 1. The speed of training can be improved by directly optimizing amortized beliefs.
\item The bound becomes tighter by using clamping of variable, clamping can be done with or without selection of variables. No sampling is needed in training or inference.
\item If need more contribution, use tree-reweighted hyper graph to obtain tighter bound.
  
\item Not necessarily done here: the bound can also be further improved by important sampling.
\end{itemize}

Reference:
\begin{itemize}
\item 1. \href{http://ssg.mit.edu/group/willsky/publ_pdfs/166_pub_AISTATS.pdf}{Wainwright, 2003, Tree-reweighted belief propagation algorithms and approximate ML estimation by pseudo-moment matching}
\item 2. \href{https://arxiv.org/abs/1510.00087}{Weller, 2015,Clamping Improves TRW and Mean Field Approximations}
\item 3. \href{https://arxiv.org/pdf/1402.0030.pdf}{Mnih, 2014, Neural Variational Inference and Learning in Belief Networks}, which describes a neural variational method for belief network. The major difference is the belief network as a DAG do not have the problem of partition function difficulty as MRF or partial observed MRF.
  
\end{itemize}

\section{Notation}
Random variable $\bm{v} \in \Xx_v$ that can be observed.
Random variable $\bm{h} \in \Xx_h$ that is hidden variable and can not be observed.

\section{Model and Problem Definition}

We define the conditional probabilistic model as
\begin{equation}\label{eq:model}
  p(\bm{v}, \bm{h} ; \bm{\theta}) = \frac{1}{Z(\bm{\theta})} \tilde{p}(\bm{v}, \bm{h} |\bm{\theta}),
\end{equation}
with

\begin{align}
  Z(\bm{\theta}) & = \sum_{\bm{v}}\sum_{\bm{h}} \tilde{p}(\bm{v}, \bm{h} |\bm{\theta})\\
  \tilde{p}(\bm{v}, \bm{h} ; \bm{\theta}) & = \exp\left\{-E(\bm{v}, \bm{h}, \bm{\theta})\right\}
\end{align}

where $E(\bm{v}, \bm{h}; \bm{\theta})$ is the average energy: $\Xx_v \times \Xx_h \rightarrow \RR$.

We want to maximize the marginal likelihood:
\begin{equation}
  \umax{\bm{\theta}}{\log\;\sum_{\bm{h}}p(\bm{v}, \bm{h}; \bm{\theta})} = \umax{\bm{\theta}}{\log\;Z(\bm{v}, \bm{\theta})} - \log\; Z(\bm{\theta}), 
\end{equation}
where $Z(\bm{v}, \bm{\theta})  = \sum_{\bm{h}}\tilde{p}(\bm{v}, \bm{h} ; \bm{\theta})$
\section{A lower bound of the marginal likelihood}
Denote $A(\bm{\theta}) = \log{Z(\bm{\theta})}$ and $A(\bm{v}, \bm{\theta}) = \log{Z(\bm{v}, \bm{\theta})}$
\begin{equation}
  E(\bm{v}, \bm{h}, \bm{\theta}) = - \langle  \bm{\theta}, \bm{\phi}(\bm{v}, \bm{h})\rangle
\end{equation}
and
\begin{equation}
  \bm{\mu} = \EE_{p(\bm{v}, \bm{h}; \bm{\theta})}[\bm{\phi}(\bm{v}, \bm{h})].
\end{equation}
In case of overcomplete representation of $\bm{\phi}$, $\bm{\mu}$ is the set of marginal distributions.


With mean field approximation,
\begin{equation}
  A_{M}(\bm{v}, \bm{\theta}) = \umax{\bm{\mu}_{\bm{v}} \in \Mm_M}{ \langle {\bm{\theta}, \bm{\mu}_{\bm{v}}} \rangle + H_{M}(\bm{\mu}_{\bm{v}})},
\end{equation}
where $\Mm_M$ is the subspace of distributions where each variable is independent. And we have
\begin{equation}
  A_{M}(\bm{v}, \bm{\theta}) \leq A(\bm{v}, \bm{\theta}).
\end{equation}

With tree-reweighted approximation, TRW,
\begin{equation}
  A_{T}(\bm{\theta}) = \umax{\bm{\mu} \in \Mm_T}{ \langle {\bm{\theta}, \bm{\mu}} \rangle + H(\bm{\mu})},
\end{equation}
where $\Mm$ is the subspace of distributions where each variable is independent. And we have
\begin{equation}
  A_{T}(\bm{\theta}) \geq A( \bm{\theta}).
\end{equation}

We define the lower bound of marginal loglikelihood:
\begin{equation}
  \Ll(\bm{\theta}) = A_{M}(\bm{v}, \bm{\theta}) - A_{T}(\bm{\theta}) \leq \log\sum_{\bm{h}}p(\bm{v}, \bm{h}; \bm{\theta}).
\end{equation}

Connection to RBM:
\begin{equation}
  p(\bm{v}, \bm{h}; \bm{\theta}) = \frac{1}{Z(\bm{\theta})} \exp\{\bm{v}\bm{W}\bm{h} + \bm{v}\bm{b} + \bm{v}\bm{a}\}
\end{equation}
Note $p(\bm{h}|\bm{h})$ is exactly independent, and thus the $A_{M}(\theta) = A(\bm{\theta})$ can be achieved, then how tight the lower bound $\Ll(\bm{theta})$ would dependent only on the TRW bound.


\color{red}{I should also consider how to use the trained model for prediction.}
\section{Experiment}
\begin{itemize}
\item Start with standard RBM \href{https://papers.nips.cc/paper/9687-amortized-bethe-free-energy-minimization-for-learning-mrfs.pdf}{section 4.2 in Amortized learning of MRFs}
  \begin{itemize}
  \item Try to break the conditinal independence by connecting nodes of $\bm{h}$
  \item Extent to conditional RBM training for denoising and data completion
  \end{itemize}
\item high-order HMMs

\end{itemize}




\chapter{Powering the expectation maximization method by neural networks}
content: \href{https://arxiv.org/abs/1907.13432}{Neural Network based Explicit Mixture Models and Expectation-maximization based Learning}, under review

section/chapter transition text: mixture model could be obtained from clamping and condition on a discrete variable, ref to Geier, \href{http://auai.org/uai2015/proceedings/papers/158.pdf}{Locally conditional belief propagation}. \href{https://papers.nips.cc/paper/5529-clamping-variables-and-approximate-inference.pdf}{Weller, clamping variables and approximate inference}

\begin{remark}
  RELATIONSHIP TO K-MEANS CLUSTERING
  Big picture: The EM algorithm for mixtures of Gaussians is like a soft version of the K-means algorithm.
\end{remark}

\begin{remark}
  EM lower bound $+$ entropy of posterior of latent variable if a free energy. ref to 10-708 lecture6 note.
  EM using posterior of latent variable is equivalent to fully observable MLE where statistics are replaced by their expectations w.r.t the posterior.
\end{remark}

Can be viewed as two-node graphical model learning. \href{https://sailinglab.github.io/pgm-spring-2019/notes/lecture-05/}{10-708lecture5-note}
\section{Normalizing flow}

\section{expectation maximization of neural network based mixture models}

\section{An alternative construction method}

\section{Experiments}


\section{Summary}

%%% Local Variables:
%%% mode: latex
%%% TeX-master: "../../main"
%%% End:

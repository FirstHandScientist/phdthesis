\chapter{Sammanfattning}

\begin{otherlanguage}{swedish}
Probabilistiska grafiska modeller utgör ett naturligt ramverk för representation av komplexa system och erbjuder enkel abstraktion av interaktioner inom systemen.
Resonemang baserade på probabilistiska grafiska modeller tillåter oss att genomföra statistisk slutledning (inferens) med osäkerhet inom ramarna för probabilistisk teori. Allmän statistisk inferens innefattar beräkning av marginella eller betingade sannolikheter för olika tillstånd i ett system, eller utvärdering av partitionsfunktionen för den underliggande fördelningen i ett slumpmässigt Markov fält (icke-riktad grafisk modell). Fördelarna med grafiska modeller beror i praktiken till stor del på tillgången till effektiva approximativa inferensmetoder som tillhandahåller snabba och träffsäkra resultat. Ett annat grundläggande problem, nära besläktat med den inferens som utförs i grafiska modeller, är att bestämma parametrarna för en grafisk kandidatmodell genom att extrahera information från empiriska observationer, d.v.s. inlärning av parametrar för en grafisk modell. Dessa två centrala uppgifter (dvs. slutledning och inlärning) både interagerar med, och underlättar för, varandra. Till exempel innefattar inlärning av en grafisk modell vanligtvis en inferensmetod som en subrutin medan den inlärda grafiska modellen sedan används för inferensproblem baserade på ny data.

I denna avhandling utvecklar vi nya algoritmer och modeller för generisk inferens i slumpmässiga Markov-fält. Inledningsvis presenterar vi en alternativ metod för sannolikhetspropagering (s.k. "belief propagation") genom minimering av divergens, vilket står i motsats till metoder baserade på minimering av fri energi. Detta alternativa tillvägagångssätt leder till utveckling av en ny typ av algoritm för sannolikhetspropagering som visar sig utgöra en generalisering av standardalgoritmen. Insikter om konvergensbeteendet hos den utvecklade algoritmen i det binära fallet tillhandahålls bortsett från intuitionen under utveckling. Som ett ytterligare steg utöver approximativ inferens baserad på s.k. "message passing" utvecklar vi en regionsbaserad energi-nätverksmodell som utför generisk inferens via regionsbaserad  minimering av fri energi. Detta transformerar slutledningsproblemet i slumpmässiga Markov-fält till ett optimeringsproblem. Denna modell innefattar vår väsentliga förståelse för både inferens och beräkningseffektivitet hos moderna neurala nätverksmodeller.

Nästa del av avhandlingen fokuserar på parameterinlärning av probabilistiska grafiska modeller. Denna del inleds med en diskussion om parameterinlärning av icke-riktade grafiska modeller samt förklarar en (approximativ) inferensmetods roll inom denna rutin. För riktade grafiska modeller presenteras nya ändliga "mixture"-modeller som inkluderar normaliseringsflöden i neurala nätverksimplementeringar för mer träffsäker och flexibel modellering. Inlärningen av de utvecklade generiska modellerna hanteras av s.k "expectation maximization" på grund av närvaron av dolda (eller latenta) variabler. Den träffsäkra modelleringsmetoden och inlärningen utvidgas ytterligare till dynamiska system inom ett reducerat dynamiskt Bayesian-nätverk, dvs en dold Markov-modell. Avhandlingen avslutas med ett kapitel om sannolikhetsfri inlärning av en viss klass av riktade grafiska modeller, där (riktade) generativa modeller genererar implicita sannolikhetsfördelningar och tränas via det optimala transportavståndet.
\end{otherlanguage}

%%% Local Variables:
%%% mode: latex
%%% TeX-master: "../main"
%%% End:

\usepackage[bookmarks]{hyperref}
\usepackage[T1]{fontenc}
\usepackage{textcomp}
\usepackage{lmodern}
\usepackage[latin1]{inputenc}
\usepackage[swedish,english]{babel}
\usepackage{amsmath,amssymb}
\usepackage{graphicx,psfrag}
\usepackage{color}
\usepackage{cite}
\usepackage{url}
\usepackage{array}
\usepackage{enumerate}
\usepackage{amsthm}
\usepackage{comment}
\usepackage{float}
\usepackage{algorithm}
\usepackage{algorithmic}
\usepackage{balance}
\usepackage{bm}
\usepackage{subfig}
\usepackage{epstopdf}
\usepackage{comment}
\usepackage{makeidx}
% Several biblographies
\usepackage{bibunits}
% figure drawing
\usepackage{tikz}
\makeindex
\usepackage{nomencl}
\makenomenclature
\renewcommand{\nomname}{Acronyms and Notations}
\usepackage{etoolbox}
\renewcommand\nomgroup[1]{
\item[\bfseries
\ifstrequal{#1}{A}{Acronyms}{%
\ifstrequal{#1}{N}{Notations}{}}
]}
\newtheorem{example}{Example}
\newtheorem{theorem}{Theorem}
\numberwithin{theorem}{chapter}
\newtheorem{lemma}{Lemma}
\numberwithin{lemma}{chapter}
\newtheorem{proposition}{Proposition}
\numberwithin{proposition}{chapter}
\newtheorem{corollary}{Corollary}
\numberwithin{corollary}{chapter}
\newtheorem{remark}{Remark}
\numberwithin{remark}{chapter}
\newtheorem{property}{Property}
\numberwithin{property}{chapter}
\newtheorem{conjecture}{Conjecture}
\numberwithin{conjecture}{chapter}
\newtheorem{assumption}{Assumption}
\numberwithin{assumption}{chapter}
\numberwithin{algorithm}{chapter}

\renewcommand{\algorithmicrequire}{\textbf{input:}}
\renewcommand{\algorithmicensure}{\textbf{output:}}


\usepackage{preamble/mystyle}


%%% Local Variables:
%%% mode: latex
%%% TeX-master: "../main"
%%% End:

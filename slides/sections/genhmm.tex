
\begin{frame}[label=current]
  {GenHMM: Bring the concept into HMM}
  \begin{tikzpicture}
    \begin{scope}[scale=0.8, local bounding box=genhmmTemplate]
      \tikzstyle{enode} = [thick, draw=black, circle, align=center]
\tikzstyle{cnode} = [thick, draw=black, circle, align=center, inner sep = 0.3pt]
\tikzstyle{nnode} = [thick, rectangle, rounded corners = 2pt,minimum size = 0.8cm,draw,inner sep = 8pt]
\node[enode] (s) at (-1.5,-1) {${s}$};
\node[enode] (k) at (-1.5,1) {${\kappa}$};
\node[enode] (z) at (-1.5,0) {$\bm{z}$};

\node[enode] (sp) at (0,-1){${s}^{\prime}$};
\node[enode] (kp) at (0,1){${\kappa}^{\prime}$};
\node[enode] (zp) at (0,0){$\bm{z}^{\prime}$};

\node[enode] (x) at (1.4,-2){$\bm{x}^{\prime}$};

\node[nnode, fit=(x)(z)(k)(s)] (box) {};
% \node[] at (-1.8,-0.4) {$\abs{\Dd}$};

\draw[->] (s) to (sp);
\draw[->] (sp) to (x);
\draw[->] (zp) to (x);
\draw[->] (kp) to (x);
%%% Local Variables:
%%% mode: latex
%%% TeX-master: "../ppgm_slide"
%%% End:

      
    \end{scope}
    \only<1>{
      \node (defText) [right=of box] {
        \begin{minipage}{0.6\linewidth}
          At time $t$, the probabilistic model of a state $s\in\Ss$ is then given by
          \vskip -0.1cm
          \begin{equation*}\label{eq:state-prob-model}
            p(\bm{x}| s; \bm{\Phi}_{s}) = \sum_{\kappa=1}^{K}\pi_{s, \kappa} p(\bm{x}| s, \kappa; \bm{\theta}_{s, \kappa}),
          \end{equation*}
          \vskip -0.1cm
          where 
          \begin{itemize}
          \item[-] $\pi_{s, \kappa} = p(\kappa | s; \bm{H})$, naturally $\sum_{\kappa = 1}^{K} \pi_{s, \kappa}= 1$
          \item[-]$p_k(\bm{x})$ is induced by the $k$th generator $\bm{g}_k(\bm{z})=\bm{g}(\bm{z};\boldsymbol{\theta}_k)$
          \end{itemize}
        \end{minipage}
      };}
    \only<2>{
      \begin{scope}[scale=0.75,every node/.append style={transform shape}, shift={($(genhmmTemplate.east)+(4, 0)$)}]
        \tikzstyle{enode} = [thick, draw=black, ellipse, inner sep = 2pt,  align=center]
        \tikzstyle{nnode} = [thick, rectangle, rounded corners = 2pt,minimum size = 0.8cm,draw,inner sep = 2pt]
        \node[enode] (z1) at (-1.2,1.8) {$\bm{z}\sim p_{s,1}(\bm{z})$};
        \node[nnode] (g1) at (1,1.8) {$\bm{g}_{s,1}$};
        \node[enode] (z2) at (-1.2,0.5){$\bm{z}\sim p_{s,2}(\bm{z})$};
        \node[nnode] (g2) at (1,0.5) {$\bm{g}_{s,2}$};
        \node[enode] (zK) at (-1.2,-1.8) {$\bm{z}\sim p_{s,K}(\bm{z})$};
        \node[nnode] (gs) at (1, -1.8) {$\bm{g}_{s,K}$};
        \node[enode, label={below:{Induced distribution}}] (px) at (4.5,0){$\bm{x}\sim p(\bm{x}| s; \bm{\Phi}_{s})$};
        
        \draw[dotted,line width=2pt] (0,-0.3) -- (0,-1.2);
        \filldraw[->] (1.9, 0.5)circle (2pt) --  node[above=0.2]{${\kappa}\sim \bm{\pi}_{s}$} (px)  ;
        \draw[->] (z1) -- (g1);
        \draw[->] (g1) -- (1.8, 1.8);

        \draw[->] (z2) -- (g2);
        \draw[->] (g2) -- (1.8, 0.5);

        \draw[->] (zK) -- (gs);
        \draw[->] (gs) -- (1.8, -1.8);
        
      \end{scope}
      \begin{scope}[on background layer, every node/.append style={transform shape}]
        \node [rounded corners = 2pt, inner sep=4pt, fill=blue!30,fit=(g1)(g2)(gs), label={[label distance=0.3cm]-60:{\tiny Mixture of generators}}] {};
      \end{scope}
    }
    \begin{scope}[scale=0.8,every node/.append style={transform shape}, shift={($(genhmmTemplate.south)+(3.5, -2cm)$)}, local bounding box=unroll]
      
\tikzstyle{enode} = [thick, draw, circle, inner sep = 3pt,  align=center]
\tikzstyle{cnode} = [thick, fill=gray, draw, circle, inner sep = 3pt,  align=center]
\node[enode] (k1) at (-2, 1) {$\kappa_1$};
\node[enode] (k2) at (0, 1) {$\kappa_2$};
\node[enode] (k3) at (2, 1) {$\kappa_3$};
\node[enode] (k4) at (4, 1) {$\kappa_4$};

\node[enode] (z1) at (-2, 0) {$\bm{z}_1$};
\node[enode] (z2) at (0, 0) {$\bm{z}_2$};
\node[enode] (z3) at (2, 0) {$\bm{z}_3$};
\node[enode] (z4) at (4, 0) {$\bm{z}_4$};

\node[enode] (s1) at (-2, -1) {$s_1$};
\node[enode] (s2) at (0, -1) {$s_2$};
\node[enode] (s3) at (2, -1) {$s_3$};
\node[enode] (s4) at (4, -1) {$s_4$};


\node[cnode] (x1) at (-1, -2) {$\bm{x}_1$};
\node[cnode] (x2) at (1, -2) {$\bm{x}_2$};
\node[cnode] (x3) at (3, -2) {$\bm{x}_3$};
\node[cnode] (x4) at (5, -2) {$\bm{x}_4$};

\draw[->] (s1) to (s2);
\draw[->] (s2) to (s3);
\draw[->] (s3) to (s4);

\draw[->] (s1) to (x1);
\draw[->] (s2) to (x2);
\draw[->] (s3) to (x3);
\draw[->] (s4) to (x4);

\draw[->] (k1) to [out=-45, in=90] (x1);
\draw[->] (k2) to [out=-45, in=90] (x2);
\draw[->] (k3) to [out=-45, in=90] (x3);
\draw[->] (k4) to [out=-45, in=90] (x4);

% \draw[->] (s2) to [out=15, in=-15] (o2);
% \draw[->] (s3) to [out=15, in=-15] (o3);
% \draw[->] (s4) to [out=15, in=-15] (o4);
\draw[->] (z1) to [out=-25, in=100] (x1);
\draw[->] (z2) to [out=-25, in=100] (x2);
\draw[->] (z3) to [out=-25, in=100] (x3);
\draw[->] (z4) to [out=-25, in=100] (x4);




%%% Local Variables:
%%% mode: latex
%%% TeX-master: "../ppgm_slide"
%%% End:

    \end{scope}
    \only<1>{
      \draw[green, ->, shorten >=10pt, shorten <=10pt] (genhmmTemplate) |- (unroll);
    }
    \only<2>{
      \draw[dashed, ->, shorten >=10pt, shorten <=10pt] (genhmmTemplate) |- (unroll);
      \begin{scope}[on background layer, every node/.append style={transform shape}]
        \node [rounded corners = 2pt, inner sep=4pt, fill=gray!30, fit=(sp)(zp)(kp)(x)] {};
      \end{scope}
    }
    \node [text width=2cm, black, left=-0.2cm of s1]{\tiny Unroll the template};
  \end{tikzpicture}
  
\end{frame}

% \begin{frame}
%   {GenHMM: Bring the concept into HMM}
%   \begin{tikzpicture}
%     \begin{scope}[scale=0.8, local bounding box=genhmmTemplate]
%       \tikzstyle{enode} = [thick, draw=black, circle, align=center]
\tikzstyle{cnode} = [thick, draw=black, circle, align=center, inner sep = 0.3pt]
\tikzstyle{nnode} = [thick, rectangle, rounded corners = 2pt,minimum size = 0.8cm,draw,inner sep = 8pt]
\node[enode] (s) at (-1.5,-1) {${s}$};
\node[enode] (k) at (-1.5,1) {${\kappa}$};
\node[enode] (z) at (-1.5,0) {$\bm{z}$};

\node[enode] (sp) at (0,-1){${s}^{\prime}$};
\node[enode] (kp) at (0,1){${\kappa}^{\prime}$};
\node[enode] (zp) at (0,0){$\bm{z}^{\prime}$};

\node[enode] (x) at (1.4,-2){$\bm{x}^{\prime}$};

\node[nnode, fit=(x)(z)(k)(s)] (box) {};
% \node[] at (-1.8,-0.4) {$\abs{\Dd}$};

\draw[->] (s) to (sp);
\draw[->] (sp) to (x);
\draw[->] (zp) to (x);
\draw[->] (kp) to (x);
%%% Local Variables:
%%% mode: latex
%%% TeX-master: "../ppgm_slide"
%%% End:

%       \draw[green,->] (sp) to (x);
%       \draw[green,->] (zp) to (x);
%       \draw[green,->] (kp) to (x);

%     \end{scope}
%     \begin{scope}[scale=0.75,every node/.append style={transform shape}, shift={($(genhmmTemplate.east)+(4, 0)$)}]
%       \tikzstyle{enode} = [thick, draw=black, ellipse, inner sep = 2pt,  align=center]
%       \tikzstyle{nnode} = [thick, rectangle, rounded corners = 2pt,minimum size = 0.8cm,draw,inner sep = 2pt]
%       \node[enode] (z1) at (-1.2,1.8) {$\bm{z}\sim p_{s,1}(\bm{z})$};
%       \node[nnode] (g1) at (1,1.8) {$\bm{g}_{s,1}$};
%       \node[enode] (z2) at (-1.2,0.5){$\bm{z}\sim p_{s,2}(\bm{z})$};
%       \node[nnode] (g2) at (1,0.5) {$\bm{g}_{s,2}$};
%       \node[enode] (zK) at (-1.2,-1.8) {$\bm{z}\sim p_{s,K}(\bm{z})$};
%       \node[nnode] (gs) at (1, -1.8) {$\bm{g}_{s,K}$};
%       \node[enode, label={below:{Induced distribution}}] (x) at (4.5,0){$\bm{x}\sim p(\bm{x}| s; \bm{\Phi}_{s})$};

%       \draw[dotted,line width=2pt] (0,-0.3) -- (0,-1.2);
%       \filldraw[->] (1.9, 0.5)circle (2pt) --  node[above=0.2]{${\kappa}\sim \bm{\pi}_{s}$} (x)  ;
%       \draw[->] (z1) -- (g1);
%       \draw[->] (g1) -- (1.8, 1.8);

%       \draw[->] (z2) -- (g2);
%       \draw[->] (g2) -- (1.8, 0.5);

%       \draw[->] (zK) -- (gs);
%       \draw[->] (gs) -- (1.8, -1.8);

%     \end{scope}
%     \begin{scope}[on background layer, every node/.append style={transform shape}]
%       \node [rounded corners = 2pt, inner sep=4pt, fill=blue!30,fit=(g1)(g2)(gs), label={[label distance=0.3cm]-60:{\tiny Mixture of generators}}] {};
%     \end{scope}

%     \begin{scope}[scale=0.8,every node/.append style={transform shape}, shift={($(genhmmTemplate.south)+(3.5, -2cm)$)}, local bounding box=unroll]
%       
\tikzstyle{enode} = [thick, draw, circle, inner sep = 3pt,  align=center]
\tikzstyle{cnode} = [thick, fill=gray, draw, circle, inner sep = 3pt,  align=center]
\node[enode] (k1) at (-2, 1) {$\kappa_1$};
\node[enode] (k2) at (0, 1) {$\kappa_2$};
\node[enode] (k3) at (2, 1) {$\kappa_3$};
\node[enode] (k4) at (4, 1) {$\kappa_4$};

\node[enode] (z1) at (-2, 0) {$\bm{z}_1$};
\node[enode] (z2) at (0, 0) {$\bm{z}_2$};
\node[enode] (z3) at (2, 0) {$\bm{z}_3$};
\node[enode] (z4) at (4, 0) {$\bm{z}_4$};

\node[enode] (s1) at (-2, -1) {$s_1$};
\node[enode] (s2) at (0, -1) {$s_2$};
\node[enode] (s3) at (2, -1) {$s_3$};
\node[enode] (s4) at (4, -1) {$s_4$};


\node[cnode] (x1) at (-1, -2) {$\bm{x}_1$};
\node[cnode] (x2) at (1, -2) {$\bm{x}_2$};
\node[cnode] (x3) at (3, -2) {$\bm{x}_3$};
\node[cnode] (x4) at (5, -2) {$\bm{x}_4$};

\draw[->] (s1) to (s2);
\draw[->] (s2) to (s3);
\draw[->] (s3) to (s4);

\draw[->] (s1) to (x1);
\draw[->] (s2) to (x2);
\draw[->] (s3) to (x3);
\draw[->] (s4) to (x4);

\draw[->] (k1) to [out=-45, in=90] (x1);
\draw[->] (k2) to [out=-45, in=90] (x2);
\draw[->] (k3) to [out=-45, in=90] (x3);
\draw[->] (k4) to [out=-45, in=90] (x4);

% \draw[->] (s2) to [out=15, in=-15] (o2);
% \draw[->] (s3) to [out=15, in=-15] (o3);
% \draw[->] (s4) to [out=15, in=-15] (o4);
\draw[->] (z1) to [out=-25, in=100] (x1);
\draw[->] (z2) to [out=-25, in=100] (x2);
\draw[->] (z3) to [out=-25, in=100] (x3);
\draw[->] (z4) to [out=-25, in=100] (x4);




%%% Local Variables:
%%% mode: latex
%%% TeX-master: "../ppgm_slide"
%%% End:

%     \end{scope}
%     \draw[dashed, ->, shorten >=10pt, shorten <=10pt] (genhmmTemplate) to [out=-120, in=180] (unroll);
%     \node [text width=2cm, black, left=-0.2cm of s1]{\tiny Unroll the template};
%   \end{tikzpicture}

% \end{frame}



\begin{frame}[label=current]{Alternative View of GenHMM}
  \centering
  \begin{tikzpicture}
    \begin{scope}[scale=0.8]
      \tikzstyle{enode} = [thick, draw=black, ellipse, inner sep = 1pt,  align=center]
      \tikzstyle{nnode} = [thick, rectangle, rounded corners = 2pt,minimum size = 0.8cm,draw,inner sep = 2pt]
      \draw[step=1.0, black , xshift=0cm,yshift=0cm] (0,0) grid (5,4);
      \draw[->] (5,4) -- node[midway, above] {$s$} (5.2,4);
      \draw[->] (0,0) -- node[midway, left] {$\kappa$} (0,-0.2);
      \node[black] at (0, 4.3) {$p(\bm{x}| s, \kappa; \bm{\Phi}_{s})$};
      % \node[enode, pin={[pin edge={->}]right:{}}] (g1) at (-0.5,1.8) {$p(\bm{x}| s=1, \kappa=1; \bm{\Phi}_{1})$};
      % \node[enode] (g2) at (-0.5,0.5) {$p(\bm{x}| s=2, \kappa=1; \bm{\Phi}_{2})$};
      % \node[enode, pin={[pin edge={->}]right:{}}] (gs) at (-0.5, -1.8) {$p(\bm{x}| s=|\Ss|, \kappa=K; \bm{\Phi}_{|\Ss|})$};
      \node[enode] (x) at (8.5,4){$\ubm{x}\sim p(\ubm{x};\bm{H})$};

      % \draw[dotted,line width=2pt] (0,-0.3) -- (0,-1.2);
      \only<1>{
        \filldraw[green, ->] (1, 1) circle (3pt) --  (x) node[green,midway,above] {$t$};
      }
      \only<2>{
        \filldraw[green, ->] (3, 3) circle (3pt) --  (x) node[green,midway,above] {$t+1$};
      }
      \only<3>{
        \filldraw[green, ->] (0, 2) circle (3pt) --  (x) node[green,midway,above] {$t+2$};
      }
      
      % \draw[->] (g1) -- (1.8, 1.8);
      % \draw[->] (g2) -- (1.8, 0.5);
      % \draw[->] (gs) -- (1.8, -1.8);

      \begin{scope}[xshift=5cm, yshift=2.5cm, thick, every node/.style={sloped,allow upside down}]
        \node[nnode] (m) at (3.5,-2) {Memory};
        \node[nnode] (a) at (3.5,-0.5) {$\bm{A}$};

        \draw (2.1,0.9)-- (2.2, 0.);
        \draw (2.2,0.)-- node {\midarrow} (2.2,-2);
        \draw (2.2,-2)-- (m);
        \draw (m)-- (5, -2);
        \draw (5, -2)-- node {\midarrow} (5 ,-0.5);
        \draw (5, -0.5) -- (a);
        \draw (a)-- node {\midarrow} (2.2, -0.5);
        \node (st) at (4.8, -1) {$s_{t}$};
        \node (st1) at (2.56, -0.25) {$s_{t+1}$};
      \end{scope}
      \begin{scope}[on background layer]
        \node [rounded corners = 4pt, rectangle, inner sep = 4pt,  align=center, fit=(a)(m)(st)(st1), label=below:{($s,\kappa$)}, fill=blue!20] () {};
      \end{scope}
      
    \end{scope}
  \end{tikzpicture}
  
  We define the hypothesis set of HMM as $\Hh := \{\bm{H} | \bm{H}=\{\Ss, \bm{q}, \bm{A}, p(\bm{x}|{s}; \bm{\Phi}_{s})\}\}$, where
  \begin{itemize}
  \item[-] $\Ss$ is the set of hidden states of $\bm{H}$.
  \item[-] $\bm{q} = \left[ q_1, q_2, \cdots, q_{|\Ss|}\right]^\intercal$ is the initial state distribution of $\bm{H}$ with $|\Ss|$ as cardinality of $\Ss$. For $i \in \Ss$, $q_i = p(s_{1}=i;\bm{H})$. We use $s_t$ to denote the state $s$ at time $t$.
  \item[-] $\bm{A}$ matrix of size $|\Ss| \times |\Ss|$ is the transition matrix of states in $\bm{H}$. That is, $\forall i, j \in \Ss$,  $\bm{A}_{i,j} = p(s_{t+1}=j|s_{t}=i; \bm{H})$.
  \item[-] For a given hidden state $s$, the density function of the observable signal is $p({\bm{x}}|{s};\bm{\Phi}_{s})$, where $\bm{\Phi}_{s}$ is the parameter set that defines this probabilistic model. Denote $\bm{\Phi} = \left\{ \bm{\Phi}_{s}| s \in \Ss \right\}$.
  \end{itemize}
\end{frame}



\begin{frame}{Learning Intuition}
  
  With empirical distribution $\hat{p}(\ubm{x}) = \frac{1}{N}\sum_{n}\delta_{\ubm{x}^{(n)}}(\ubm{x})$, learning of GenHMM boils down to
  \begin{equation*}
    \min_{\bm{H}\in \Hh}\;\mathrm{KL}(\hat{p}(\ubm{x})\|p(\ubm{x};\bm{H}))
  \end{equation*}
  Then problem becomes maximum likelihood estimation \\
  \begin{equation*}
    \hat{\bm{H}} =    \arg \max_{\bm{H} \in \Hh} \log \textstyle\prod_{i} p(\ubm{x}^{i};\bm{H}),
  \end{equation*}

  \begin{itemize}
  \item E-step: 
    the $\Qq$ function
    \begin{equation*}\label{eq:em-q-funciton}
      \Qq(\bm{H}; \bm{H}^{\mathrm{old}}) = \underbrace{\EE_{\hat{p}(\ubm{x}),p(\ubm{s},\ubm{\kappa}| \ubm{x}; \bm{H}^{\mathrm{old}})}}_{\text{\begin{tabular}{c}w.r.t. distribution $\hat{p}(\ubm{x})$ \\and $p(\ubm{s},\ubm{\kappa}| \ubm{x}; \bm{H}^{\mathrm{old}})$\end{tabular}}} \underbrace{\left[\log\,p(\ubm{x}, \ubm{s}, \ubm{\kappa}; \bm{H})\right]}_{\text{\begin{tabular}{c}complete lklh.\\ factorize over time\end{tabular}}},
    \end{equation*}
    
  \item M-step: the maximization step
    \begin{equation*}\label{eq:em-m-opt}
      \umax{\bm{H}} \Qq(\bm{H}; \bm{H}^{\mathrm{old}}).
    \end{equation*}

    The M-step can be reformulated as
    \begin{align}\label{eq:m-step-subs}
      \umax{\bm{H}} \Qq(\bm{H}; \bm{H}^{\mathrm{old}})
      =\underbrace{\umax{\bm{q}}\Qq(\bm{q}; \bm{H}^{\mathrm{old}})}_\text{Initial State}
      + \underbrace{\umax{\bm{A}}\Qq(\bm{A}; \bm{H}^{\mathrm{old}})}_\text{Transition} 
      + \underbrace{\umax{\bm{\Phi}}\Qq(\bm{\Phi}; \bm{H}^{\mathrm{old}})}_\text{Generators},
    \end{align}
    Maximizing a lower bound of $\log \textstyle\prod_{i} p(\ubm{x}^{(i)};\bm{H})$.

  \end{itemize}
  \only<2->{
    \begin{textblock}{5}(3,6)
      \begin{tikzpicture}[auto,rotate=-5,transform shape]
        % \node [black] (a) at (0,0) {
        % \textcolor{blue}{em block, message passing or backward-forward}
        % };
        \node[text width=7cm, align=left] (asent) at (6,0)
        {\begin{minipage}{1\textwidth}
            \begin{itemize}[label=\textbullet]
            \item $F$ = $\Qq$ + Entropy
            \item E-step require inference (message-passing)
            \item No optimality in M-step (NN generators, bash-size gradient descent).
            \item Still, guaranteed non-decreasing lklh. (c.f. Proposition 7.1)
            \end{itemize}
          \end{minipage}
          % $\begin{aligned}
          %   F &:= \Ee + H
          %   \mathrm{E~step:}~&\max_{q} F(q,\bm{\Phi}^{\mathrm{old}}) \\
          %   \mathrm{M~step:}~&\max_{\bm{\Phi}} F(q^{\mathrm{old}},\bm{\Phi})
          % \end{aligned}$
        };
        \begin{scope}[on background layer]
          \node [rounded corners = 4pt, rectangle, inner sep = 4pt,  align=center, fit=(asent), label=below:{}, fill=\stampcolor] () {};
        \end{scope}
        % \draw[-] (-3, 0.6) to (8, 0.6);


      \end{tikzpicture}
    \end{textblock}
  }
\end{frame}



\begin{frame}{Application of GenHMM}
  Speech Recognition: \\
  \centering
  Configuration GenHMM in experiments on TIMIT
  \vskip -0.2cm
  \begin{table}
    \begin{adjustbox}{width=0.65\textwidth}
      \begin{tabular}{lc}
        \toprule
        \begin{tabular}[x]{@{}c@{}} Latent distribution $p_{s,\kappa}(\bm{z})$ \\ $s\in \Ss, \kappa=1, 2, \cdots, K $ \end{tabular} & Standard Gaussian \\
        \hline
        Number of flow blocks & $4$ \\
        \hline
        Non-linear mapping $\bm{m}_a$, $\bm{m}_b$ & \begin{tabular}[x]{@{}c@{}} Multiple layer perception \\ $3$ layers and with hidden dimension $24$ \end{tabular}\\
        \bottomrule
      \end{tabular}
    \end{adjustbox}
  \end{table}

  \begin{columns}
    \column{0.5\textwidth}
    \centering
    {Phoneme classification / recognition}
    \vskip -0.2cm
    \begin{table}
      \centering
      \begin{adjustbox}{width=0.8\textwidth}
        \begin{tabular}{llccc}
          \toprule
          {Model} & Criterion &  K=1 &  K=3 &  K=5  \\  \midrule
          \multirow{3}{*}{\begin{tabular}{c}GMM-HMM \\ \tiny linear variable change\end{tabular}}
                  & Accuracy    & 62.3 &  68.0 &  68.7  \\
                  & {Precision} & 67.9 &  72.6 &  73.0  \\
                  & {F1}        & 63.7 &  69.1 &  69.7 \\
          \midrule
          \multirow{3}{*}{\begin{tabular}{c}GenHMM \\ \tiny non-linear variable change\end{tabular}}
                  & Accuracy    & 76.7   & \textbf{77.7} &  {77.7} \\ 
                  & {Precision} & 76.9   & \textbf{78.1} &  78.0 \\
                  & {F1}        & 76.1   & \textbf{77.1} &  77.0\\
          \bottomrule                                                                  
        \end{tabular}
      \end{adjustbox}
    \end{table}
    \column{0.5\textwidth}
    \centering
    Robustness to perturbation of noise.
    \vskip -0.2cm
    \begin{table}
      \centering
      \begin{adjustbox}{width=0.90\textwidth}
        \begin{tabular}{llcccc}
          \toprule
          \multirow{2}{*}{Model} & \multirow{2}{*}{Criterion} &
                                                                \multicolumn{4}{c}{White Noise SNR} \\
                                                                % \cline{3-6}
          
                                 && {15dB} &  {20dB} &  {25dB} & {30dB}  \\
          \midrule
          \multirow{3}{*}{GMM-HMM}
                                 & Accuracy & 36.6 &  44.2 &  50.8 & 57.1
          \\
                                 &Precision & 59.2 &  64.2 &  68.4 & 70.6  \\
                                 & F1       & 39.9 &  47.7 &  53.9 & 59.9 \\
          \midrule
          \multirow{3}{*}{GenHMM}
                                 & Accuracy & 52.4 & 62.0 &  69.7 & \textbf{74.3} \\
                                 &Precision & 60.0 &  65.9 &  71.7 & \textbf{74.8}  \\
                                 & F1       & 52.5 &  62.0 &  69.3 & \textbf{73.5} \\
          \bottomrule                                                                  
        \end{tabular}
      \end{adjustbox}
      \vspace{0.1cm}
    \end{table}
  \end{columns}
  
  \flushleft
  \vskip -0.5cm
  Application to sepsis detection for infants, c.f. see Section 7.5.
  \begin{itemize}[label=\textbullet]
  \item generative training + discriminative training
  \item innovative feature engineering: bring derivative features into this problem
  \end{itemize}
  

\end{frame}

%%% Local Variables:
%%% mode: latex
%%% TeX-master: "../ppgm_slide"
%%% End:


{ \setbeamercolor{background canvas}{bg=hl_bg}
  \setbeamercolor{normal text}{fg=hl_fg}
  \setbeamercolor{frametitle}{fg=hl_fg}
  \begin{frame}
    \usebeamercolor[fg]{normal text}
    \begin{center}
      {
        \begin{tikzpicture}
          \tikzstyle{cnode} = [thick, draw=white, ellipse, inner sep = 2pt,  align=center]
          \tikzstyle{fnode} = [thick, draw=white, ellipse, inner sep = 10pt,  align=center]
          \tikzstyle{rnode} = [thick, rectangle, inner sep = 1.5pt,  align=left]
          \node[rnode] (inf) at (-2, 0) {\large Inference};
          \node[rnode, below = 0.6cm of inf.west, anchor=west] (abp) {$\bullet$ {$\alpha$-BP}};
          \node[rnode, below = 1.2cm of inf.west, anchor=west] (renn) {$\bullet$ RENN};
          \node[cnode, fit=(abp)(inf)(renn)] (infn) {};
          
          \node[rnode, right = 3 of inf] (lern) {\large Learning};
          \node[rnode, below = 0.4 of lern.west, anchor=west] (genmm) {$\bullet$ GenMM};
          \node[rnode, below = 0.8 of lern.west, anchor=west] (genhmm) {{$\bullet$} GenHMM};
          \node[rnode, below = 1.2 of lern.west, anchor=west] (lfree) {{$\bullet$} EOTGM};
          \node[cnode, fit=(lern)(genmm)(genhmm)(lfree)] (learn) {};

          \node[fnode, fit=(infn)(lern)] (box) {};

          
          \node[below right = 0.5 and -0.5 of infn] {{Probabilistic} Graphical Model};
          \draw[->,line width=0.5mm, green] (infn) to[out=15, in=165] (learn);
          \draw[->,line width=0.5mm, green] (learn) to[out=195, in=-15] (infn);
        \end{tikzpicture}
      }
    \end{center}
    
  \end{frame}
}
% \begin{frame}{Two Direction Impact}
%   Why impact in two direction?
%   \begin{itemize}[label=$\bullet$]
    
%     \item Learning to Inference:\\
%       \begin{figure}[!t]
%         \centering
%         \begin{tikzpicture}
%           \tikzstyle{cnode} = [thick, draw=black, ellipse, inner sep = 2pt,  align=center]
%           \tikzstyle{fnode} = [thick, draw=black, ellipse, inner sep = 10pt,  align=center]
          
%           \node[cnode] (infn) at (0,0) {Inference};
%           \node[cnode] (lern) at (3,0) {Learning};
          
%           % \node[fnode, fit=(infn)(lern)] (box) {};
%           % \node[] at (1.4, -0.6) {Probabilistic Graphical Model};
%           \draw[->,line width=0.2mm] (infn) to[out=15, in=165] (lern);
%           \draw[green, ->,line width=0.2mm] (lern) to[out=195, in=-15] (infn);
%         \end{tikzpicture}
%       \end{figure}
%       Stuff we have been taken as default, a given $p(\bm{x};\bm{\theta})$
%       \begin{itemize}[label=$\bullet$]
%       \item built by expert knowledge, or
%       \item built by extracting information from evidence (empirical data).
%       \end{itemize}
%     \item Inference to Learning:\\
%       \begin{figure}[!t]
%         \centering
%         \begin{tikzpicture}
%           \tikzstyle{cnode} = [thick, draw=black, ellipse, inner sep = 2pt,  align=center]
%           \tikzstyle{fnode} = [thick, draw=black, ellipse, inner sep = 10pt,  align=center]
          
%           \node[cnode] (infn) at (0,0) {Inference};
%           \node[cnode] (lern) at (3,0) {Learning};
          
%           % \node[fnode, fit=(infn)(lern)] (box) {};
%           % \node[] at (1.4, -0.6) {Probabilistic Graphical Model};
%           \draw[green, ->,line width=0.2mm] (infn) to[out=15, in=165] (lern);
%           \draw[->,line width=0.2mm] (lern) to[out=195, in=-15] (infn);
%         \end{tikzpicture}
%       \end{figure}
%       Model learning: an error trial process that compares inferred 'fact' and actual fact (evidence).\\
%       Model learning usually needs inference as a subroutine, which sometimes are replaced by sampling in particle based methods.
    
%   \end{itemize}
  
  
% \end{frame}
\begin{frame}{\large Inference Routine in Learning}
  
  What is $\bm{\theta}$ in $p(\bm{x}; \bm{\theta}) = \frac{1}{Z(\bm{\theta})} \prod_{a} \psi_a(\bm{x}_a; \bm{\theta}_a)$? \\
  A direct view:
  \begin{align*}
    \max_{\bm{\theta}} \log{p(\bm{x}; \bm{\theta})} =  \max_{\bm{\theta}}\sum_{a}\log{ \psi_a(\bm{x}_a; \bm{\theta}_a) } \underbrace{- \textcolor{blue}{\log{Z(\bm{\theta})}}}_{\text{can be est. by $\min{F_V}$}},
  \end{align*}
  An alternative view:
  \begin{align*}
    \pd{\log{p}(\bm{x};\bm{\theta})}{\bm{\theta}_a} = \pd{\log{{\phi_a}(\bm{x}_a; \bm{\theta}_a)}}{\bm{\theta}_a} - \underbrace{\EE_{\textcolor{blue}{p(\bm{x}_a; \bm{\theta})}}\left[ \pd{\log{{\phi_a}(\bm{x}_a; \bm{\theta}_a)}}{\bm{\theta}_a} \right]}_{\text{can be est. by beliefs}}.
  \end{align*}

  Remark:
  \begin{itemize}[label={$\bullet$}]
  \item This essentially requires estimation of partition function or marginal probabilities.
  \item Stationary points translate into moment matching.
  \end{itemize}
  \only<2->{
  \begin{textblock}{5}(3,10)
    \begin{tikzpicture}[auto,rotate=-5,transform shape]
      \tikzstyle{cnode} = [thick, draw=black, ellipse, inner sep = 2pt,  align=center]
      \tikzstyle{fnode} = [thick, draw=black, ellipse, inner sep = 10pt,  align=center]
          
      \node[cnode] (infn) at (0,0) {Inference};
      \node[cnode] (lern) at (3,0) {Learning};
      
      \draw[->,line width=0.2mm] (infn) to[out=15, in=165] (lern);
      \draw[->,line width=0.2mm] (lern) to[out=195, in=-15] (infn);

      \node[text width=7cm, align=left, below =of infn.west, anchor=west] (uncouple)
      {\begin{minipage}{1\textwidth}
          \begin{itemize}[label=\textbullet]
          \item Two modules are not necessarily coupled
          \item Each module may be replaced by another algorithm while the other one remains.
          \end{itemize}
        \end{minipage}
      };
      \begin{scope}[on background layer]
        \node [rounded corners = 4pt, rectangle, inner sep = 4pt,  align=center, fit=(uncouple)(infn)(lern), label=below:{}, fill=\stampcolor] () {};
      \end{scope}
    \end{tikzpicture}
  \end{textblock}
}
\end{frame}

% \begin{frame}[label=current]{\large Inference Routine in Learning}
%   \begin{tikzpicture}
%     \node[black] (grad) at (0,0) {
%     \begin{minipage}{1\linewidth}
%       What is $\bm{\theta}$ in $p(\bm{x}; \bm{\theta}) = \frac{1}{Z(\bm{\theta})} \prod_{a} \psi_a(\bm{x}_a; \bm{\theta}_a)$? \\
%       A direct view:
%       \begin{align*}
          %           \max_{\bm{\theta}} \log{p(\bm{x}; \bm{\theta})} =  \max_{\bm{\theta}}\sum_{a}\log{ \psi_a(\bm{x}_a; \bm{\theta}_a) } \underbrace{- \log{Z(\bm{\theta})}}_{\text{Helmholtz free energy, can be est. by $F$}},
          %         \end{align*}
          %           An alternative view:
          %           \begin{align*}
          %           \pd{\log{p}(\bm{x};\bm{\theta})}{\bm{\theta}_a} = \pd{\log{{\phi_a}(\bm{x}_a; \bm{\theta}_a)}}{\bm{\theta}_a} - \underbrace{\EE_{p(\bm{x}_a; \bm{\theta})}\left[ \pd{\log{{\phi_a}(\bm{x}_a; \bm{\theta}_a)}}{\bm{\theta}_a} \right]}_{\text{can be est. by beliefs}}.
          %         \end{align*}

          %           Remark:
          %           \begin{itemize}[label={$\bullet$}]
          %           \item This essentially requires estimation of Helmholtz free energy or marginal probabilities.
          %           \item Stationary points translate into moment matching.
          %           \end{itemize}
          %           \end{minipage}
          %           };
          %           \pause {
          %           \begin{scope}[local bounding box=uncouple]
          %           \tikzstyle{cnode} = [thick, draw=black, ellipse, inner sep = 2pt,  align=center]
          %           \tikzstyle{fnode} = [thick, draw=black, ellipse, inner sep = 10pt,  align=center]

          %           \node[cnode] (infn) at (0,0) {Inference};
          %           \node[cnode] (lern) at (3,0) {Learning};
          %           \node[black, above=of infn.north] (txt1) {at The two modules are not necessarily coupled};
          %         %           \node[fnode, fit=(infn)(lern)] (box) {};
          %         %           \node[] at (1.4, -0.6) {Probabilistic Graphical Model};
          %           \draw[->,line width=0.2mm] (infn) to[out=15, in=165] (lern);
          %           \draw[->,line width=0.2mm] (lern) to[out=195, in=-15] (infn);
          %           \node[black,fill=white, below=of infn.south] (txt2) {Each module may be replaced by another algorithm while the other one remains.};
          %           \begin{scope}[on background layer]
          %           \node[rectangle, inner sep = 10pt,  align=center, fit=(txt1)(txt2)(infn)(learn), fill=white] {};
          %           \end{scope}

          %           \end{scope}

          %           }  
          %           \end{tikzpicture}
          %           \end{frame}




\begin{frame}
  {Learning MRFs}
  {What is $\bm{\theta}$ in $p(\bm{x};\bm{\theta})$?}
  \vskip 0.5cm
  Table of negative log-likelihood of learned MRFs \\
  \begin{table}
    \begin{adjustbox}{width=0.9\textwidth}
      \begin{tabular}{lcccccccc}
        % std=1.0
        \toprule
        N & True & Exact & Mean Field & Loopy BP & Damped BP & GBP & Inference Net & RENN \\
        \toprule
        \multicolumn{9}{c}{Grid Graph}\\
        \midrule
        25  &  9.000  &  9.004  &  9.811  &  {9.139}  &  9.196  &  10.56  &  9.252  &  \textbf{9.048}  \\
        100 &  19.34  &  19.38  &  23.48  &  {19.92}  &  20.02  &  28.61  &  20. 29  &  \textbf{19.76} \\
        225 &  63.90  &  63.97  &  69.01  &  66.44    &  66.25  &  92.62  &  68.15  &  \textbf{64.79}  \\
        \toprule
        % std=1.0
        \multicolumn{9}{c}{Complete Graph}\\
        \midrule
        9  &  3.276  &  3.286  &  9.558  &  5.201  &  5.880  &  10.06  &  5.262  & \textbf{3.414}  \\
        16  &  4.883  &  4.934  &  28.74  &  13.64  &  18.95  &  24.45  &  13.77  &  \textbf{5.178}  \\

        \bottomrule
      \end{tabular}
    \end{adjustbox}
  \end{table}
  \vskip 0.5cm
  {Average consumed time per epoch (unit: second) for two MRF learning cases.}
  \begin{table}
    \centering
    \begin{adjustbox}{width=0.9\textwidth}
      \begin{tabular}{lcccccc}
        \toprule
        {} &  Mean Field & Loopy BP & Damped BP & GBP & Inference Net & RENN \\
        \toprule
        Grid $\Gg$,\! $N\!=\!225\!\!$ & 40.09 & 335.1 & 525.1 & 12.37 & 19.49 & 16.03 \\
        Complete\! $\Gg$,\! $N=\!16\!\!$ & 2.499 & 12.40 & 5.431 & 1.387 & 0.882 & 2.262 \\
        \bottomrule
      \end{tabular}
    \end{adjustbox}
  \end{table}

\end{frame}

%%% Local Variables:
%%% mode: latex
%%% TeX-master: "../ppgm_slide"
%%% End:
